%-----------------------------------------------
%  Engineer's & Master's Thesis Template
%  Copyleft by Artur M. Brodzki & Piotr Woźniak
%  Warsaw University of Technology, 2019-2022
%-----------------------------------------------

\documentclass[
    bindingoffset=5mm,  % Binding offset
    footnoteindent=3mm, % Footnote indent
    hyphenation=true    % Hyphenation turn on/off
]{src/wut-thesis}

\graphicspath{{tex/img/}} % Katalog z obrazkami.
\addbibresource{bibliografia.bib} % Plik .bib z bibliografią

%-------------------------------------------------------------
% Wybór wydziału:
%  \facultyeiti: Wydział Elektroniki i Technik Informacyjnych
%  \facultymeil: Wydział Mechaniczny Energetyki i Lotnictwa
% --
% Rodzaj pracy: \EngineerThesis, \MasterThesis
% --
% Wybór języka: \langpol, \langeng
%-------------------------------------------------------------
\facultymeil    % Wydział Elektroniki i Technik Informacyjnych
\MasterThesis % Praca inżynierska
\langeng % Praca w języku polskim

\begin{document}

%------------------
% Strona tytułowa
%------------------
\instytut{Aeronautics and Applied Mechanics}
\kierunek{Robotics and Automatic Control}
\specjalnosc{Robotics}
\title{
    Recognition of objects in the environments \\ 
    of autonomous vehicles in real-time
 
}
% Title in English for English theses
% In English theses, you may remove this command
\engtitle{
    Unnecessarily long and complicated thesis' title \\
    difficult to read, understand and pronounce
}
% Title in Polish for English theses
% Use it only in English theses
\poltitle{
    Rozpoznawanie obiektów w otoczeniu pojazdów autonomicznych 
w czasie rzeczywistym
}
\author{Iskel Fikiru Hordofa}
\album{323715}
\promotor{Prof. Rzymkowski Cezary }
\date{\the\year}
\maketitle

%-------------------------------------
% Streszczenie po polsku dla \langpol
% English abstract if \langeng is set
%-------------------------------------
\cleardoublepage % Zaczynamy od nieparzystej strony
\abstract The study aims to develop a deep learning-based obstacle avoidance system for mobile robots using Neural Architecture Search (NAS) in Python. By leveraging advanced computer vision techniques, the goal is to equip mobile robots with the ability to perceive their environment, detect obstacles in real-time, and navigate safely through complex surroundings. The paper gives a balanced assessment of the performance of YOLO-NAS, highlighting its optimized efficiency in achieving an optimal balance between accuracy and speed, thus surpassing other human-designed models in terms of efficiency. Additionally, it emphasizes the adaptability of YOLO-NAS to diverse tasks and hardware, positioning it as a valuable tool across a spectrum of applications. The strategic pre-training of YOLO-NAS equips the model with a comprehensive understanding, enhancing its capabilities for real-time object detection applications. 

\keywords Object detection, Neural Network, Deep Learning, YOLO, YOLO-NAS

%----------------------------------------
% Streszczenie po angielsku dla \langpol
% Polish abstract if \langeng is set
%----------------------------------------
\clearpage
\secondabstract Streszczenie. Celem pracy jest analiza systemu unikania przeszkód przez pojazdy autonomiczne, w tym roboty mobilne, opartego na metodach głębokiego uczeniu się złożonych algorytmów sztucznych sieci neuronowych. Wykorzystanie zaawansowanych technik wizji komputerowego w wyposażeniu pojazdów autonomicznych, zapewnia zdolność postrzegania i analizy otoczenia, wykrywania przeszkód w czasie rzeczywistym i bezpiecznego poruszania się w złożonym otoczeniu. Praca przedstawia ocenę wydajności modelu sieci YOLO-NAS i jego wariantów, podkreślając jego wysoką wydajność w zakresie osiągania optymalnej równowagi pomiędzy dokładnością i szybkością przy użyciu tych samych hiperparametrów, w czym przewyższa inne modele zaprojektowane przez człowieka. Przeprowadzona analiza dodatkowo podkreśla możliwość dostosowania modelu YOLO-NAS do różnorodnych zadań i sprzętu, pozycjonując go jako cenne narzędzie o szerokim spektrum zastosowań. Strategia wstępnego treningu YOLO-NAS wyposaża ten model w „umiejętność” wszechstronnego zrozumienie rozwiązywanego zadania, zwiększając jego możliwości w zastosowaniach do wykrywania obiektów w czasie rzeczywistym.

\secondkeywords Rozpoznawanie obiektów, sztuczne sieci neuronowe, 
głębokie uczenie maszynowe, YOLO, YOLO-NAS
\clearpage
\includegraphics[width=\textwidth]{src/en/authorship/eiti-1.pdf}
\clearpage
\includegraphics[width=\textwidth]{src/en/authorship/eiti-2.pdf}

% \pagestyle{plain}
%--------------
% Spis treści
%--------------
\cleardoublepage % Zaczynamy od nieparzystej strony
\tableofcontents

%------------
% Rozdziały
%------------
\cleardoublepage % Zaczynamy od nieparzystej strony
\pagestyle{headings}
\section{Introduction}

The field of mobile robots is expanding quickly and has the potential to completely transform a variety of industries. Mobile robots are becoming increasingly common in our daily lives as the need for automation rises. These robots are capable of assisting from healthcare to industrial automation. The advent of autonomous vehicles has revolutionized the transportation industry, promising safer and more efficient travel. However, the realization of this vision hinges on the development of advanced technologies capable of enabling these vehicles to navigate their environment safely and efficiently. One such technology is real-time object recognition, which allows autonomous vehicles to identify and classify objects in their path. 

Autonomous vehicles rely heavily on sensors and computer vision algorithms to perceive their surroundings. These systems must be able to process large amounts of data in real time to make quick decisions, often within milliseconds. The challenge lies in developing algorithms that can accurately recognize objects while maintaining high computational efficiency.  Obstacle avoidance, which is essential for autonomous navigation, is one of the major problems in mobile robots. In the realm of mobile robotics, obstacle avoidance stands as a fundamental challenge, empowering robots to traverse cluttered environments with both safety and efficiency \cite{wenzel2021vision}.\\
 Traditional obstacle avoidance methods, such as sonar, laser scanning, and stereo vision, have served as the cornerstone of these systems. However, these methods have inherent limitations, particularly in terms of their ability to handle dynamic and unpredictable scenarios.

Recently, the integration of deep learning, a subset of machine learning, has brought about a paradigm shift in the field of autonomous vehicles. Deep learning algorithms, particularly Convolutional Neural Networks (CNNs), have proven to be highly effective in tasks such as image and speech recognition, natural language processing, and beyond. In the context of autonomous vehicles, deep learning has been instrumental in enhancing the perception of surroundings, driver behavior monitoring, path planning, sensor fusion, and intelligent control of vehicle dynamics.

Deep learning-based control algorithms are especially crucial for achieving full autonomy. They can manage complex lateral and longitudinal maneuverings and can effectively calculate steering commands for lateral control and acceleration and braking commands for longitudinal speed control of vehicle 1. However, the safety of deep neural networks can be unstable under adverse conditions, necessitating the use of varied datasets for training the automated control system.

This paper is focused on the application of deep learning algorithm called YOLO-NAS in real-time object recognition for autonomous vehicles. 







\section{Problem Statement}
Developing real-time object detection systems for obstacle avoidance represents a nuanced and intricate task, necessitating meticulous consideration. The efficacy of such systems is contingent upon several pivotal factors, encompassing the quality of visual data, the intricacies of the underlying algorithms, and the overall architecture of the obstacle avoidance system. A thorough exploration of each of these components is imperative to optimize the system for superior performance.

In addressing this inquiry, a comprehensive examination of the influencing factors is undertaken, seeking avenues for their refinement and enhancement. The goal is to engineer an obstacle-avoidance system of the highest quality. Furthermore, it is essential to contemplate the broader implications of the obstacle avoidance solution, assessing how it contributes to heightened safety and enhanced mobility across diverse contexts. By prioritizing these fundamental aspects, a profound understanding of the challenges inherent in real-time object detection systems can be cultivated. This nuanced understanding serves as a foundation for the development of innovative solutions, leveraging advanced computer vision techniques to surmount these obstacles.

\section{General Objective:}
The general objective of this study is to develop a deep learning-based obstacle avoidance system for mobile robots using Neural Architecture Search(NAS) in Python. The aim is to enable mobile robots to perceive their environment, detect obstacles in real time, and navigate safely through complex surroundings in real time.
\subsection{Specific Objectives:}

\begin{itemize}
     \item Collect and curate a dataset of labeled images for training the obstacle detection neural network.
    \item Preprocess the dataset by applying techniques such as image augmentation and normalization to improve the network's performance.
    \item Fine-tune and implement a neural network architecture suitable for obstacle detection and avoidance.
    \item 	Integrate the trained neural network into the mobile robot's control system for real-time obstacle detection and avoidance
    \item Evaluate the performance of the obstacle avoidance system using quantitative metrics such as accuracy, precision, recall, and F1 score.
    \item Compare the neural network-based approach with traditional methods of obstacle avoidance in terms of effectiveness, efficiency, and adaptability.
    \item Assess the robustness and generalization capabilities of the trained neural network by testing it in various environments and with different types of obstacles.
    \item Identify potential areas for further improvement and research in computer vision-based obstacle avoidance, such as multi-sensor fusion or incorporating depth information.
    \item Provide practical insights and recommendations for implementing computer vision-based obstacle avoidance systems for mobile robots using neural networks in Python.
\end{itemize}

\section{General Theory}
\subsection{Image processing}
Image processing is a method of performing operations on an image to extract information or enhance its features. It is a broad field that covers a range of techniques, including image acquisition, Pre-processing, segmentation, feature extraction, image analysis, and visualization. In general, image processing involves transforming an image into a more useful form for analysis or display. This can be achieved through a variety of techniques, such as filtering, edge detection, Thresholding, and morphological operations. Image processing has numerous applications in various fields, including medicine, remote sensing, surveillance, robotics, and entertainment. \cite{wilhelm2016digital}, \cite{tyagi2018understanding} 

\subsection{Deep Learning and Neural Networks}
Deep learning is another name for a multilayer artificial neural network or multilayer perceptron. The elementary
bricks of deep learning are the neural networks, that are combined to form the deep neural networks.

\begin{table}[htbp]
\centering
\caption{Comparing Neural Networks and Deep Learning}
\footnotesize
\begin{tabular}{ |p{3cm}|p{5cm}|p{5cm}| }%{ |c|c|c| }
\hline
Feature & Neural Networks & Deep Learning \\
\hline
Structure & Single-layer or shallow neural networks (typically two to three layers) & Deep neural networks with multiple layers (typically more than three) \\
\hline
Learning Ability & Can learn simple patterns and make predictions & Can learn complex patterns, including abstract relationships, and make more accurate predictions \\
\hline
Data Requirements & Requires less training data & Requires more training data to learn complex patterns patterns \\
\hline
Computational Cost & Less computationally expensive & More computationally expensive, especially for deep neural networks with many layers \\
\hline
Applications & Image classification, fraud detection, spam filtering & Image recognition, natural language processing, speech recognition, medical diagnosis, self-driving cars \\
\hline
\end{tabular}
\end{table}

The term "deep" in deep learning refers to the use of multiple layers in the neural network. Deep learning models, often called deep neural networks (DNNs), are capable of learning and representing complex patterns and hierarchical features from data. \cite{nielsen2015neural} \cite{Arnold2011AnIT}, \cite{2015Natur.521..436L} \\

Various types of deep learning systems exist, distinguished by their neural network architectures and operational principles. Examples include feed-forward neural networks, convolutional networks, recurrent neural networks, autoencoders, and deep beliefs. These methodologies have facilitated substantial advancements in sound and image processing domains, encompassing tasks such as facial recognition, speech recognition, computer vision, automated language processing, and text classification (e.g., spam detection). The potential applications of these techniques are diverse. An illustrative instance is the AlphaGo program, which achieved world champion status in the game of Go in 2016 by leveraging deep learning methodologies.\cite{koons2005}, \cite{shinde2018review}, \cite{choi2020introduction}
\begin{figure}[H]
    \centering
    \includegraphics[width=0.8\linewidth]{tex/img/NN_DL_ML.png}
    \caption{Neural network core of deep learning \cite{an}}
    \label{fig:NN_DL}
\end{figure}

\subsection{Neural Networks}
Neural networks are the fundamental components of deep learning. They are composed of layers of interconnected nodes or neurons that process information.
Artificial neural networks (ANNs) are computing systems that are designed to work in a similar way to the human brain.

\subsubsection{The Biological Inspiration}
The complexity of the brain has made it challenging for scientists to fully understand it, despite extensive research. Engineers have modified neural models to create a more useful and less biological approach, while still maintaining much of the original terminology.
\begin{figure}[H]
    \centering \includegraphics[width=0.5\linewidth]{tex/img/Structure_of_Neurons.PNG}
    \caption{The structure of Neurons \cite{nielsen2015neural}}
    \label{fig:Neuron structure}
\end{figure}
A neuron is composed of three main parts - the cell body, the dendrite (which branches out to receive input), and the axon (which branches out to send output). Connections between axons and dendrites are made via synapses. Electro-chemical signals are carried from the dendrites, through the cell body, and down the axon to other neurons.


\subsubsection{Perceptron}
A single neuron of a neural network is called a perceptron. A perceptron implements a mathematical function that operates on the input signals and generates outputs. Figure is an example of a perceptron. A perceptron is the simplest neural network.
\subsubsection{The learning process}
A perceptron model starts by multiplying all the input values and their weights and then sums these values to create a weighted sum. This weighted sum is then applied to the activation function "f" to obtain the desired result. This activation function is also known as a step function and is represented by "f". This step function or activation function plays an essential role in ensuring that the output signal is mapped between the required values (0,1) or (-1,1). Please note that the weight of the input data indicates the strength of the node. similarly, the value of the input bias allows the activation function curve to shift up or down.
The input signals to a neuron come either from a source (a camera or sensing device) or from the outputs of other neurons.

\begin{figure}[H]
    \centering
    \includegraphics[width=0.7\linewidth]{tex/img/Perceptron.png}
    \caption{Perceptron}
    \label{fig:Perceptron}
\end{figure}

Mathematically, the output (y) of a neuron in a neural network perceptron can be expressed as:\\

\(y=f(\sum_{i=1}^{n}(\omega_{i}.x_{i})+b)\)

where:\\
$\omega_{i}$ is the weight associated with the input \(x_{i}\)\\
\(b\) is the bias term,\\
\(f:\) activation\\
As shown in fig \ref{fig:Perceptron}, One percepteron has the following components: \\
\begin{itemize}
    \item \textbf{Input Nodes or Input Layer: }  
    The input layer, also called input nodes, is the first layer of a neural network. It is responsible for receiving input data, which can be any numerical values, images or text. The input layer does not perform any calculations or transformations of the input data, but only forwards them to subsequent layers in the network. \cite{ansari2020building}, \cite{nielsen2015neural}\\
    \item \textbf{Weight: } Weights are parameters that adjust the strength of the connections between neurons (nodes) in adjacent layers of a neural network. Each connection has an associated weight, and these weights are learnable and updated during the training process. The weights determine the impact of the input signals on the neurons in the next layer. A higher weight means that the corresponding input has a stronger influence on the neuron's output. During training, the neural network adjusts these weights to minimize the difference between the predicted output and the actual target output. \\
    \item \textbf{Bias: } Biases are additional parameters in a neural network that allow for fine-tuning the output of each neuron. Unlike weights, biases are not associated with specific inputs but are added to the weighted sum of inputs in each neuron.
    Role: Biases provide the neural network with flexibility, enabling it to account for situations where all inputs are zero or have low values. They allow neurons to activate even when the weighted sum of inputs is not sufficient to trigger an output.Like weights, biases are adjusted during training to improve the overall performance of the network. \cite{ansari2020building}, \cite{nielsen2015neural}\\
    \item \textbf{Activation Function: } The primary purpose of an activation function is to determine whether a neuron should be activated (output a signal) or not, based on the input it receives.
    An activation function is a mathematical operation enacted on the output of a neuron (or node) within the architecture of a neural network. It introduces non-linearity to the network, allowing it to learn complex patterns and make more sophisticated decisions. The activation function takes the weighted sum of inputs and a bias term and produces the output of the neuron.\cite{nielsen2015neural}

\end{itemize}
\subsubsection{Types of Perceptron: }
Based on the layers, Perceptron models are divided into two types.
\begin{enumerate}
    \item \textbf{Single-layer perceptron model:}  This is one of the simplest types of artificial neural networks (ANN). The single-layer perceptron model consists of a feedforward network and also includes a threshold transfer function inside the model. The main goal of the single-layer perceptron model is to analyze linearly separated objects with binary outputs.

    In a single-layer perceptron model, its algorithms do not contain recorded data, so it starts with a non-constantly allocated input for weight parameters. Furthermore, it sums all inputs (weight). If, after adding all inputs, the sum of all inputs is greater than the set value, the model will be activated and display the output value as +1.
    If the result is the same as the previously established threshold value, then the performance of this model is considered satisfactory and the weight requirements do not change. However, this model contains several discrepancies that arise when many input weight values are introduced into the model. Therefore, some changes should be necessary when entering the weights to find the desired result and minimize errors.
    A single-layer perceptron learns patterns that can be solved linearly. \cite{ansari2020building}, \cite{knerr1990single}

        \item \textbf{Multi-layer Perceptron} An artificial neural network, similar to the human brain, comprises several neurons, also known as perceptrons. A cluster of neurons processes the inputs. Every individual neuron within the group separately processes the inputs. The outputs from this cluster of neurons are transmitted to another individual neuron or cluster of neurons for subsequent processing. These neurons can be visualized as organized in layers, where the output of one layer serves as the input for the following layer. There is no limit to the number of layers you can use to train your neural network. The arrangement of neurons in a neural network, where multiple layers are used, is referred to as a multilayer perceptron (MLP). A multi-layer perceptron model shares the same structure as a single-layer perceptron model, but it includes a higher number of hidden layers.
    \begin{figure}[H]
        \centering
        \includegraphics[width=0.8\linewidth]{tex/img/MLP.PNG}
        \caption{Multi layer perceptron \cite{ansari2020building}}
        \label{fig:MLP}
    \end{figure}
also known as the backpropagation algorithm, which executes in two stages as follows:
    \begin{itemize}
        \item \textbf{Forward Stage:} Activation functions start from the input layer in the forward stage and terminate at the output layer.
        \item \textbf{Backward Stage:} In the backward stage, weight and bias values are modified as per the model's requirements. In this stage, the error between the actual output and demand originates at the output layer and propagates backward to the input layer.
    \end{itemize}
A multi-layer perceptron model is a type of artificial neural network that has greater processing power and is capable of processing both linear and non-linear patterns. It is also capable of implementing logic gates such as AND, OR, XOR, NAND, NOT, XNOR, and NOR.

This model has multiple artificial neural networks with several layers, where the activation function is not linear as in a single-layer perceptron model. Instead, it can be executed with various activation functions such as sigmoid, TanH, ReLU, etc., for deployment. \cite{ansari2020building}, \cite{nielsen2015neural}
\end{enumerate}

\subsection{Deep Learning}
Deep learning is a type of artificial neural network or multilayer perceptron. It is a subset of machine learning that focuses on using artificial neural networks and representational learning. The term "deep" in deep learning refers to the use of multiple layers in the network. These methods can be supervised, semi-supervised, or unsupervised. \cite{vargas2017deep}

Various deep-learning architectures, such as deep neural networks, deep belief networks, deep reinforcement learning, recurrent neural networks, convolutional neural networks, and transformers, have been successfully applied to several fields, including computer vision, voice recognition, signal processing, natural language processing (NLP), machine translation, bio-informatics, drug or medical purposes, climate forecasting, material inspection, and board game programs. These architectures have produced remarkable results that are comparable to, and in some cases, surpassing human expert performance. \cite{hosseini2020deep}

\subsubsection{Deep Learning or Multilayer Perceptron Architecture}
A multilayer perceptron consists of at least three types of layers: the input layer, the hidden layers, and the output layer. You can have more than one hidden layer. Each layer contains one or more neurons. A neuron performs some computation on the inputs it gets and generates outputs. The output from the neurons is sent as input to the next layer, except in the case of the output layer.

\begin{enumerate}
    \item \textbf{Input layer: } This is the primary component of Perceptron, which accepts the initial data into the system for further processing. Every input node in the system holds a numerical value that is real and can be expressed as a number.
In a neural network, the input layer receives raw input data like image, text, etc. It then passes it on to the next layer. Each node in the input layer represents a feature or an attribute of the input data. These nodes act as receptors for the raw input information and are connected to the nodes in the next layer, which is typically a hidden layer.
    
Here are some key points about the input layer:
    \begin{enumerate}
\item Nodes/Neurons: The nodes in the input layer are also called input neurons. Each neuron corresponds to a specific feature or dimension of the input data.
\item Input Features: If you're working with a dataset of, for example, images, each node in the input layer might represent a pixel's intensity or color values. In the case of text data, each node could represent a unique word or a character.
\item No Processing: The nodes in the input layer do not perform any processing on the input data. They simply pass the raw input values to the next layer. The actual computation and learning occur in the subsequent layers, particularly in the hidden layers.
\item Connections: Each node in the input layer is connected to every node in the next layer (usually a hidden layer) through weights. These weights determine the strength of the connection between nodes and play a crucial role in the learning process of the neural network.
    \end{enumerate}
    The input layer of a neural network consists of neurons that are equal in number to the features of the data. Along with these neurons, additional nodes called bias nodes can also be included in each layer. The primary purpose of the bias node is to provide control over the output of the layer. Although not strictly necessary in deep learning, it is still a common practice to add a bias node.
    
    \textbf{Total Number of Neurons in the input layer} = The number of input features without a bias = (The number of input features + 1) with a bias 
    \cite{ansari2020building}\\
    \item \textbf{Hidden Layers: } Hidden layers in a neural network refer to the layers that come between the input layer and the output layer. They are called "hidden" because their outputs are not directly observable or part of the final network output during the training phase. At least one hidden layer is necessary for a neural network to function as this is where the learning occurs. The neurons in this layer perform the computations required for learning. Generally, for most cases, a single hidden layer is sufficient for learning. However, as required to model real-world situations, the number of hidden layers can be increased. As the number of hidden layers increases, the computation complexity of the network also increases, which leads to an increase in computation time.
    \\
    
    \textbf{Total Number of Neurons in the hidden layer} = A common practice is to take two-thirds (or 66 percent) of the number of neurons in the previous layer.\\
    
    For example, if the number of neurons in the input layer is 100, the number of neurons in the first hidden layer will be 66 and in the next hidden layer will be 43, and so on. Again, there is no magic number, and you should tune the neuron counts based on the model accuracy.
    \item \textbf{Output Layer: } The output layer is the final layer in a neural network, responsible for producing the model's predictions or outputs based on the learned representations from the preceding layers. The number of neurons in the output layer depends on the problem type that the neural network is supposed to solve. \\
    \textbf{Nodes in the Output Layer can be used for:}
    \begin{itemize}
        \item \textbf{For binary classification: } there is typically one node with a sigmoid activation function, producing a probability output.\\
        \item \textbf{For multi-class classification: } when the network has to predict one of many classes, the output layer has as many neurons as the number of all possible classes.\\
        \item \textbf{For regression tasks: } there is usually a single node with a linear activation function, producing a continuous output/ when the network has to predict a continuous value, such as the closing price of stocks, the output node has only one neuron. \cite{calin2020deep}, \cite{nielsen2015neural}
    \end{itemize}
\end{enumerate}
\subsubsection{Forward propagation}

A feedforward neural network is a type of artificial neural network comprising neurons interconnected in a manner that avoids cyclic connections. This architecture represents the most basic form of the neural network, where data propagation occurs unidirectionally, from the input layer through the hidden layer(s), and ultimately to the output layer. Unlike its counterparts, this network lacks any loopback or feedback mechanism, enabling a linear flow of information through its layers. \cite{coates2011analysis} \cite{ansari2020building} 
\begin{figure}[H]
    \centering
    \includegraphics[width=0.7\linewidth]{tex/img/Forwardpropagation.PNG}
    \caption{Forward Propagation \cite{ajitjaokar}}
    \label{fig:enter-label}
\end{figure}
a single pass of forward propagation translates mathematically to:
\[Prediction=A(A(XW_{h})W_{o})\]
Where A:    is an activation function like ReLU, X is the input and $W_{h}$ and $W_{o}$ are weights.\\

\textbf{Steps}\\
- Calculate the weighted input to the hidden layer by multiplying X by the hidden weight $W_{h}$.\\
- Apply the activation function and pass the result to the final layer \\
- Repeat step 2 except this time X is replaced by the hidden layer’s output, H \cite{ansari2020building}, \cite{nielsen2015neural}

\subsubsection{BackPropagation: }
The main objective of back propagation is to modify or adjust the weights in the neural network in a way that corresponds to their contribution to the overall error. By consistently reducing the error associated with each weight, we can eventually obtain a set of weights that generate accurate predictions. \\
Here are the final 3 equations that together form the foundation of backpropagation. \cite{hecht1992theory} \cite{nielsen2015neural}
 \[
 Output Layer Error    E_{o} = (O-y).R'(Z_{o})
 \]
 \[ 
 Hidden Layer Error    E_{h} = E_{o}.W_{o}.R'(Z_{h})
 \]
 \[ 
 Cost-Weights Deriv \hspace{3mm}   LayerError.LayerInputs
 \]


\begin{figure}[H]
    \centering
    \includegraphics[width=0.8\linewidth]{tex/img/backward_propagation.PNG}
    \caption{Backward Propagation \cite{ajitjaokar}}
    \label{fig:backward_propagation}
\end{figure}
\subsubsection{Activation function}
The activation function plays a crucial role in determining whether a neuron should be activated (turned on or off) based on whether its input is relevant for model prediction. This function normalizes the output of each neuron to a range between 0 and 1 or between -1 and 1. Different mathematical functions are utilized as activation functions for various purposes.\cite{sharma2017activation}, \cite{apicella2021survey}, \cite{bfortuner_mlglossary}

we can broadly divide the activation function into two: \\
\begin{itemize}
    \item \textbf{Linear Activation Function: } Commonly used in the output layer for regression tasks where a continuous range of values is desired. A linear activation function computes a weighted sum of its input without introducing non-linearity, and The output is a linear transformation of the input. 
    \begin{figure}[H]
        \centering
        \includegraphics[width=0.5\linewidth]{tex/img/Linear_function.PNG}
        \caption{Linear Activation Function}
        \label{fig:LAF}
    \end{figure}
    \textbf{Equation: }
    \(f(x)=cx\) \\where c is constant.\\
    
    The output of the linear activation function varies from $-\infty$ to  $+\infty$, as shown in Figure above.
     If you choose to use a linear activation function, the last layer of your neural network will simply be a linear function of the first layer, regardless of how many layers the network has. This means that your network can only learn linear dependencies between the input and output, which is insufficient for solving complex problems like computer vision. Therefore, using a linear activation function is not recommended for such problems. \cite{bfortuner_mlglossary}\\
     \item \textbf{Non-linear Activation Function: } A non-linear activation function introduces non-linearity into the network, enabling it to learn complex relationships and representations. It facilitates the model's ability to generalize and differentiate outputs with diverse data.
    \begin{figure}[H]
        \centering\includegraphics[width=0.5\textwidth]{Non-Linear.PNG}
        \caption{Non-Linear Activation Function}
    \end{figure}
    Essential for training deep neural networks as it allows the network to capture complex patterns and relationships in the data, They allow the network to capture intricate relationships between features, which is crucial for tasks like image recognition, natural language processing, and more.
    \begin{enumerate}
        \item \textbf{Sigmoid or Logistic Activation Function: } 
    The sigmoid activation function calculates the neuron output using the sigmoid function, as shown here:
    $$\phi(z)=\frac{1}{1+e^{-z}}$$
    \begin{figure}[H]
        \centering\includegraphics[width=0.5\textwidth]{sigmoid_Activation.PNG}
        \caption{Sigmoid Activation Functioin}
    \end{figure}
    where \(z\) is calculated like \\
    
    $$z=X_0+\sum_{i=0}^{i=n}w_{i}x_{i}$$
    
    The sigmoid function is a mathematical operation that always yields a value between 0 and 1. This makes the output spontaneous, without many overrides as the input value fluctuates. Additionally, the sigmoid function is very useful as it does not generate a constant value from the first-order derivatives because it is a non-linear function.
    
    The sigmoid function is a mathematical function that always produces an output value between 0 and 1. This characteristic results in a smooth output without many jumps as the input value fluctuates. Another benefit of the sigmoid function is that it is nonlinear, and its first-order derivative does not generate a constant value.
    \item \textbf{Tanh or hyperbolic tangent Activation Function}
    
    The Tanh function is a type of activation function, commonly used in neural networks. Its behavior is similar to the sigmoid activation function, but with some notable differences. The Tanh function is zero-centered, and The range of the tanh function is from (-1 to 1).
    
    \begin{figure}[H]
        \centering\includegraphics[width=0.5\textwidth]{Tangent_activation.PNG}
        \caption{Tangent Activation Functioin}
    \end{figure}
    
    The TanH activation function calculates the neuron output using:\\
    $$tanh(z)=\frac{e^{z}-e^{-z}}{e^{z}+e^{-z}}$$
    The TanH is useful when it comes to mapping the values; the negative inputs will be mapped strongly to negative, and the same for positive inputs too, and the zero inputs will be mapped near zero.
    \item \textbf{ReLU (Rectified Linear Unit) Activation Function}
   The Rectified Linear Unit (ReLU) is a widely used activation function in deep learning, particularly for computer vision tasks. ReLU is preferred for image processing because it can handle non-negative values, which is a common characteristic of image pixels. One of the key advantages of ReLU is its computational efficiency, which makes it suitable for large-scale models. Furthermore, ReLU is a nonlinear function with a derivative, allowing for backpropagation and weight adjustment during training. This property is essential for the neural network to learn from the data and improve its performance over time.
    
    \begin{figure}[H]
        \centering\includegraphics[width=0.5\textwidth]{Relu_Activation.PNG}
        \caption{ReLU Activation Functioin}
    \end{figure}
    
    
    $$f(x)=max(0,x)$$
    it ranges from 0 to +$\infty$\\
    \item  \textbf{Leaky ReLU Activation Function}
    The leaky rectified linear unit (ReLU) is a widely used activation function in deep learning models. The function introduces a small negative slope in the negative region, which allows for backpropagation for negative inputs, thereby avoiding the vanishing gradient problem. However, a drawback of this function is that it produces inconsistent outputs for negative values. This is because the negative slope of the activation function results in a non-zero output for negative inputs, which is not consistent with the expected behavior of an activation function. Nonetheless, the leaky ReLU remains a popular choice for deep neural networks due to its ability to improve the training of deep models.
    \begin{figure}[H]
        \centering\includegraphics[width=0.5\textwidth]{leaky_relu.png}
        \caption{Leaky ReLU Activation Functioin}
    \end{figure}
    
    $f(y)=\left\{\begin{array}{rcl}
         \alpha y & \mbox{for} & y<0\\
         y & \mbox{for} & y\geq 0
    \end{array}\right.$\\
    \item \textbf{SELU Actiovation Function}
    A scaled exponential linear unit (SELU) computes neuron outputs using the following equation:
    
    $f(\alpha,x)=\lambda\left\{\begin{array}{rcl}
         \alpha (e^{x}-1) & \mbox{for} & x<0\\
          x & \mbox{for} & x\geq 0
    \end{array}\right.$\\
    \\
    where the value of $\lambda$ = 1.05070098 and the value of $\alpha$ = 1.67326324. These values are fixed and do not change during backpropagation.[Orielly]
    \begin{figure}[H]
        \centering\includegraphics[width=0.5\textwidth]{SELU_activation.PNG}
        \caption{SELU Activation Functioin}
    \end{figure}
    
   SELU is an activation function that has the property of self-normalization. This means that with SELU, the entire network is self-normalizing, which makes it efficient in terms of computation and helps it converge faster. Additionally, SELU overcomes the issues of exploding or vanishing gradients that occur when the input features are too high or too low.
    \item \textbf{Softplus Activation Function}
    The softplus activation function applies smoothing to the activation function value 
    \(z\) . It uses the log of exponent as follows:\\
    
    $$f(x)=lon(1+e^{z})$$\\
    \\
    Softplus is also called the SmoothReLU function. The first derivation of the softplus function is  $\frac{1}{1+e^{z}}$, which is the same as the sigmoid activation function. 
    
    \begin{figure}[H]
        \centering\includegraphics[width=0.5\textwidth]{Softplus_activation.PNG}
        \caption{Softplus Activation Function}
    \end{figure}
    \item \textbf{Softmax}
    The Softmax function operates on a vector of real numbers by normalizing it to create a probability distribution that generates outputs between 0 and 1. The resulting probabilities have the property that the sum of all output values is equal to 1. This function is typically utilized as the activation function for the output layer of a classification neural network. The interpreted output values represent prediction probabilities for each class.
    \begin{figure}[H]
        \centering\includegraphics[width=0.5\textwidth]{Softmax.PNG}
        \caption{Softmax Activation Function}
    \end{figure}
    $$\sigma(z)_{i}=\frac{e^{z_{i}}}{\sum_{j=1}^{k}e^{z_{j}}}$$ for \(i\)=1,...., k and z=$(z_{1}....,z_{k})$ $\epsilon R^{k}$
    \end{enumerate}
\end{itemize}

\subsubsection{Loss Function/Error Function}
A loss function is a function that measures how well a neural network models the training data by comparing the target and predicted output values. During training, the goal is to minimize the loss between the predicted and target outputs.
The equation of error may be written in a simplified form as follows:\\

\(Error = Expected outcome - Predicted outcome\)\\

When a neural network begins the learning process, it initializes weights and calculates output from each neuron using an activation function. It then computes the error, adjusts the weights, recalculates outputs, and re-evaluates errors, until it reaches the minimum error. The weights that give the minimum errors are considered the final weights and the network is considered "learned" at this stage.

In calculus, if the first derivative of a function is zero, then the function at that point is either a minimum or a maximum. The neural network training process aims to find the minimum point where the first derivative is zero. To achieve this, a neural network must have an \textbf{error function} that calculates the first derivative and identifies the points (weights and biases) where the error function is minimum. The type of error function used depends on the type of model being trained. These error functions are also called loss functions or simply losses.\\

The error functions are broadly divided into the following three categories: \cite{ansari2020building}, \cite{heaton2018ian}\\
\begin{itemize}
    \item \textbf{Regression loss functions} are used when we want to train models to predict continuous value outcomes, typically a numerical value. To measure the difference between the predicted and actual target values, different loss functions are employed. The following are some widely used regression loss functions:\\
    \begin{itemize}
    \item \textbf{Mean Squared Error (MSE) Loss:} This is the default error function for regression problems. This is the preferred loss function if the distribution of the target variable is normal or Gaussian. This function has numerous properties that make it especially suited for calculating loss. The difference is squared, which means it does not matter whether the predicted value is above or below the target value; however, values with a large error are penalized.\\
    $$MSE=\frac{1}{n}\sum_{i=1}^{n}(y^{(i)}-\hat{y}^{(i)})^{2}$$
    \item  \textbf{Mean Absolute Error (MAE) Loss:}  MAE finds the average of the absolute differences between the target and the predicted outputs.
    $$MAE=\frac{1}{n}\sum_{i=1}^{n}|y^{(i)}-\hat{y}^{(i)}|$$
    In some cases, this loss function serves as an alternative to MSE. As mentioned earlier, MSE is highly sensitive to outliers, which can significantly affect the loss due to the squared distance. To mitigate this, MAE is used when the training data has a substantial number of outliers.
    \item \textbf{The Mean Squared Logarithmic Error (MSLE)} The Mean Squared Logarithmic Error (MSLE) is a loss function that is commonly used in regression tasks. It is particularly useful when the target values span multiple orders of magnitude. MSLE measures the mean squared difference between the natural logarithm of the predicted values and the natural logarithm of the true values. This can be especially helpful when there is a wide variation in the scale of the target values.
    The formula for MSLE is as follows:
     $$MSLE(y,\hat{y})=\frac{1}{n}\sum_{i=1}^{n}(log(1+y^{(i))}-log(1+\hat{y}^{(i)}))^{2}$$
     where: \\
     $y^{i}: $ is the true target value for the \(i\)-th sample.\\
     $\hat{y}^{(i)}: $ is the predicted target value for the \(i\)-th sample.
     \item \textbf{Huber Loss: } The Huber loss, also referred to as the smooth L1 loss, is a commonly used loss function in regression tasks. It is designed to be less sensitive to outliers compared to the Mean Squared Error (MSE) loss function while retaining the benefits of a quadratic loss function for smaller errors.

     $$Huber(y,\hat{y})=\frac{1}{n}\sum_{i=1}^{n}L_{\delta}(y^{(i))}-\hat{y}^{(i)})$$
     where: \\
     $y_{i}: $ is the true target value for the \(i\)-th sample.\\
      $y^{i}: $ is the predicted target value for the \(i\)-th sample.\\
      $n: $ is the total number of samples.\\
      $L_{\delta}(x)$ is defined as:
      
      $L_{\delta}(x)=\left\{\begin{array}{rcl}
           \frac{1}{2}x^{2} & \mbox{for} & |x| \leq \delta \\
          \delta(|x|-\frac{1}{2}\delta) & otherwise 
      \end{array}\right.$\\
      \textbf{Benefit: } One big When training neural networks, using Mean Absolute Error (MAE) can cause a problem due to its large gradient, which can result in missing the minima at the end of training when using gradient descent. In contrast, as the loss gets closer to its minima, the gradient decreases when using Mean Squared Error (MSE), making it more accurate.
    
    To address this issue, Huber loss can be very useful since it curves around the minima, decreasing the gradient. Additionally, Huber loss is more resistant to outliers than MSE. Therefore, it combines the desirable properties of both MSE and MAE.
    \item  \textbf{Log-Cosh loss: } The Log-Cosh loss is a smooth and differentiable approximation of the Huber loss, often used in regression tasks. Similar to the Huber loss, it aims to be less sensitive to outliers than the Mean Squared Error (MSE) loss, but it has the advantage of being continuously differentiable.
     $$Log-Cosh(y,\hat{y})=\frac{1}{n}\sum_{i=1}^{n}Log(cosh(y^{(i))}-\hat{y}^{(i)}))$$
     where: \\
      $y_{i}: $ is the true target value for the \(i\)-th sample.\\
      $y^{i}: $ is the predicted target value for the \(i\)-th sample.\\
      $n: $ is the total number of samples.\\
      \(Cosh: \) is the hyperbolic cosine function.\\
      The optimization goal during training is to minimize the Log-Cosh loss, and the model adjusts its parameters to achieve predictions that result in smaller Log-Cosh loss values. The Log-Cosh loss is often considered when a balance between the robustness of the Huber loss and the differentiability of the MSE loss is desired.
      \item \textbf{Quantile loss function: } Quantile loss is a loss function used in quantile regression, where the goal is to predict not just a central tendency (like the mean in traditional regression) but rather different quantiles of the target distribution. It is particularly useful when you want to estimate a range of possible values for a prediction.
      $$L_{\gamma}(y, y^{p})=\sum_{i=y_{i}<y_{i}^{p}}(\gamma-1).|y_{i}-{y_{i}^{p}}|+ \sum_{i=y_{i}\geq y_{i}^{p}}(\gamma).|y_{i}-y_{i}^{p}|$$

      The optimization goal during training is to minimize the Quantile Loss, and the model adjusts its parameters to achieve predictions that capture the desired quantiles of the target distribution. Quantile regression is particularly useful in scenarios where understanding the variability or uncertainty in predictions is essential.
      
    
     
    \end{itemize}
    \item \textbf{Binary classification loss functions} are used when we want to train models to predict a maximum of two classes (usually denoted as 0 and 1) and the true binary label, such as to compare and contrast different classes together. \\
          \begin{itemize}
          \item \textbf{Binary Crossentropy Loss (Log Loss): } The Binary Crossentropy Loss (Log Loss) is the default loss function for binary classification problems and is considered better than other functions. It calculates a score that reflects the average difference between the actual and predicted probability distributions for predicting class 1. This score is minimized, and a perfect cross-entropy value is set to 0. The function can only be used when the target value is within the range of (0,1).
          $$Binary Crossentropy(y, \hat{y})=\frac{1}{n}\sum_{i=1}^{n}[y_{i}log(\hat{y_{i}})+(1-y_{i})log(1-\hat{y_{i}})]$$

          It measures the cross-entropy between the true binary labels and the predicted probabilities. It penalizes deviations from the true labels by assigning higher penalties to confidently wrong predictions.
          \item \textbf{Hinge Loss (SVM Loss):} This is used mainly in support of vector machine–based binary classification and can be used when the target variable is in the range (-1, 1).
          $$Hinge-Loss(y,\hat{y})= \frac{1}{n}\sum_{i=1}^{n}max(0,1-y_{i}.\hat{y_{i}})$$

          Penalizes misclassifications linearly and encourages correct predictions to have a margin of at least 1.

          \item \textbf{Squared Hinge Loss: } The squared hinge loss function computes the squared value of the score hinge loss. This mathematical operation smooths the surface of the error function, rendering it numerically more tractable.
          \[
          \text{Squared Hinge Loss}(y,\hat{y})=\frac{1}{n}\sum_{i=1}^{n}max(0,1-y_{i}.\hat{y_{i}})^{2}
          \]
      \end{itemize}
    \item \textbf{Multiclass classification loss functions} are used when our models need to predict more than two classes/are used to measure the difference between predicted class probabilities and true class labels in scenarios where there are more than two classes, such as object detection. \\
          \begin{itemize}
          \item \textbf{Categorical Crossentropy Loss: } used for multiclass classification tasks with one-hot encoded true class labels. It is used to measure the cross-entropy between the true distribution and the predicted class probabilities.
          $$Categorical Crossentropy(y,\hat{y_{i}})=-\frac{1}{n}\sum_{i=1}^{n}\sum_{j=1}^{m}y_{ij}log(\hat{y_{ij}})$$
          \item \textbf{Sparse Categorical Crossentropy Loss: } Similar to categorical crossentropy but used when true class labels are provided as integers rather than one-hot encoded vectors.
          $$Sparse Categorical Crossentropy(y,\hat{y})=-\frac{1}{n}\sum_{i=1}^{n}log(\hat{y_{i}[y_{i}]})$$
          Sparse cross-entropy performs the same cross-entropy calculation of error without requiring that the target variable be one hot-encoded before training and is used When you have a large number of classes in the target, for example, predicting dictionary words.
          \item \textbf{Kullback-Leibler divergence (KLD) loss: }The Kullback-Leibler divergence (KLD) loss is a measure of the dissimilarity between one probability distribution and a reference baseline distribution. A value of 0 for the KL divergence loss indicates that both distributions are identical. This statistical metric quantifies the amount of information loss, expressed in bits, when the predicted probability distribution is employed to approximate the target probability distribution.

        The KLD loss is a valuable tool for addressing complex problems, such as auto-encoders, which are utilized for learning high-dimensional representations. In the context of multiclass classification, KLD serves as a multiclass cross-entropy measure.
        \[
        \text{KL Divergence}(P \parallel Q) = \sum_{i \in \chi} P(i) \log\left(\frac{P(i)}{Q(i)}\right)
        \]
        where:\\
        For discrete probability distributions, P and Q are defined on the same sample space, $\chi$, the relative entropy from Q to P.
        \item \textbf{Cross-Entropy Loss (Sigmoid Crossentropy for Multilabel Classification): }
        Suitable for multilabel classification where each sample can belong to multiple classes. It measures the crossentropy for each class independently.

         \[
         \text{Cross Entropy Loss}(y,\hat{y})=-\frac{1}{n}\sum_{i=1}^{n}[y_{i}log(\sigma(\hat{y_{i}}))+(1-y_{i})log(1-\sigma(\hat{y_{i}}))]
         \]
      \end{itemize}

\end{itemize}

\subsubsection{Layers}
In a neural network, layers are the building blocks that organize and structure the computation. Each layer contains a group of nodes, or neurons, that process information.
\begin{itemize}
    \item \textbf{Convolutional Layer: } In convolutional neural networks (CNN), the convolution layer performs a linear operation by multiplying the input with a weight (kernel or filter), and it plays a critical role in the network. The layer comprises two primary components: \\
    \begin{itemize}
        \item \textbf{Kernel (Filter): } A convolution layer can comprise more than one filter. The size of the filter should be smaller than the input dimension intentionally, as it allows the filter to be applied at different positions on the input. Filters are useful for identifying significant features in a given input. By applying more than one filter to the same input, different features can be extracted. The output from multiplying the filter with the input creates a dimensional array called the "feature map."

        The Stride property controls the movement of the filter over the input. When the value is set to 1, the filter moves one column at a time over the input. When set to 2, the filter jumps two columns at a time as it moves over the input.
    \end{itemize}
    \item \textbf{Dropout Layer: } A dropout layer is a type of layer commonly used in neural networks to prevent overfitting. It takes the output of the previous layer's activations and randomly sets a certain fraction, known as the dropout rate, of the activations to 0. This effectively cancels or "drops out" those activations, helping to prevent the network from becoming too reliant on any one feature or pattern. The dropout rate is a tunable hyperparameter that can be adjusted to measure performance with different values. Typically, it is set between 0.2 and 0.5, but it can be set arbitrarily depending on the specific application and dataset.
    
    Dropout is a technique used during training of neural networks. During training, some of the activations in a layer are randomly dropped or turned off. However, at test time, no activations are dropped, instead, they are scaled down by a factor of the dropout rate. This is to account for the fact that more units are active during test time than during training time. The idea behind dropout is to introduce some noise into the layer in order to disrupt any interdependent learning or coincidental patterns that may occur between units in the layer that are not significant. This helps to prevent overfitting and improve the generalization performance of the network. \\
    \item \textbf{Pooling layer: }Pooling layers often take convolution layers as input. A complicated dataset with many objects will require a large number of filters, each responsible for finding patterns in an image so the dimensionally of a convolutional layer can get large. It will cause an increase in parameters, which can lead to over-fitting. Pooling layers are a type of technique that is used to reduce the dimensionality of high-dimensional data. Similar to convolutional layers, pooling layers also have a kernel size and stride. The kernel size is typically smaller than the feature map, with a standard size of 2x2 and a stride of 2. There are two main types of pooling layers that are commonly used.

    The first type is the max pooling layer. The Max pooling layer will take a stack of feature maps (convolution layer) as input. The value of each node in the max pooling layer is calculated by taking the maximum value of the pixels contained in a sliding window. This operation helps to reduce the size of the feature maps while preserving the most important information.
    
    The other type of pooling layer is the: \\
    \textbf{Average Pooling layer}. The average pooling layer calculates the average of pixels contained in the window. It's not used often but you may see this used in applications for which smoothing an image is preferable.\\
    \item \textbf{Fully-connected/Linear Layer: } A fully-connected layer, also called a linear layer, is a type of layer in a neural network where all the inputs from one layer are connected to every activation unit of the next layer. Usually, the last few layers in machine learning models are fully-connected ones, which output a class prediction based on the features learned in the previous layers.

To input a vector of nodes activated in the previous convolutional layers, the fully-connected layer passes it through one or more dense layers before sending it to the output layer. An activation function is used to make a prediction before the vector reaches the output layer. While the convolutional and pooling layers tend to use a ReLU function, the fully-connected layer can use two activation functions depending on the classification problem:

- Sigmoid:  is a mathematical function that is commonly used for binary classification problems. It is a logistic function that has a characteristic "S" shaped curve.
- Softmax: A more generalized logistic activation function that ensures the values in the output layer sum up to 1. It is commonly used for multi-class classification.

The activation function outputs a vector with the same dimensions as the number of classes to be predicted. The output vector yields a probability between 1 and 0 for each class. \cite{schmidhuber2015deep}, \cite{goodfellow2016deep}\\
   
\subsubsection{Optimizer}
An optimizer is an algorithm or method used to adjust the parameters of the neural network (weights and biases) during the training process. The primary goal of an optimizer is to minimize the loss function, which measures the difference between the predicted output and the true target values.

During training, the neural network makes predictions, and the optimizer adjusts the model's parameters based on the error (loss) between these predictions and the actual target values. The optimization process involves finding the optimal set of parameters that minimize the loss, enabling the neural network to make accurate predictions on unseen data. \cite{choi2019empirical}, \cite{ansari2020building} 

Some commonly used optimizers in neural networks include:
\begin{itemize}
    \item \textbf{Adaptive gradient (Adagrad): } Adaptive gradient (Adagrad) is a technique that adjusts the learning rate according to a specific parameter. 
\begin{itemize}
    \item This method ensures that parameters with higher gradients or frequent updates have a slower learning rate, so as not to overshoot the minimum value.
    \item parameters with low gradients or infrequent updates have a faster learning rate, allowing them to be trained quickly.
    \item  It divides the learning rate by the sum of squares of all previous gradients of the parameter.
    \item When the sum of the squared past gradients has a high value, it basically divides the learning rate by a high value, so the learning rate will become less.
    \item Similarly, if the sum of the squared past gradients has a low value, it divides the learning rate by a lower value, so the learning rate value will become high.
    \item This implies that the learning rate is inversely proportional to the sum of the squares of all the previous gradients of the parameter.
    \end{itemize}
    \[
    g_{t}^{i}=\frac{\partial J(w_{t}^{i})}{\partial \mathbf{W}}
    \]
    \[
    \mathbf{W}=\mathbf{W}-\alpha\frac{\partial J(\omega_{t}^{t})}{\sqrt{\sum_{r=1}^{t}(g_{r}^{i})^2+\epsilon}}
    \]
    where: \\
     $g_{t}^{i}$  the gradient of a parameter\\
    $\alpha$ : the learning rate\\
    $\epsilon: $ very small value to avoid dividing by zero
    
    \item \textbf{Adaptive delta(Adadelta): } Adadelta is a type of stochastic gradient descent algorithm that offers adaptive techniques for hyperparameter tuning. The name Adadelta is derived from "adaptive delta", where delta refers to the difference between the current weight and the newly updated weight.

    Adadelta is an improved version of Adagrad that adjusts learning rates based on a moving window of gradient updates, instead of accumulating all past gradients. This allows Adadelta to continue learning even after many updates have been made.
    
    The update rule in Adadelta eliminates the need to set a default learning rate, making it unnecessary to specify a learning rate.
    
    \[
    v_{t}=\rho v_{t-1}+(1-\rho)\bigtriangledown_{\theta}^{2}J(\theta)
    \]
    \[
    \bigtriangleup\theta=\frac{\sqrt{\omega_{t}+\epsilon}}{\sqrt{v_{t}+\epsilon}}\bigtriangledown_{\theta}J(\theta)
    \]
    \[
    \theta=\theta-\eta\bigtriangleup\theta
    \]
    $\omega_{t}=\rho\omega_{t-1} + (1-\rho)\bigtriangleup\theta^{2}$
    \item \textbf{Adam Optimizer: } The Adam optimizer combines concepts from both RMSProp and Momentum to help in computing adaptive learning rates for each parameter. It works as follows:
    \begin{itemize}
        \item Firstly, it computes the exponentially weighted average of past gradients $(v_{dW})$
        \item  Secondly, it calculates the exponentially weighted average of the squares of past gradients $(s_{dW})$
        \item Thirdly, to counteract any bias towards zero, a bias correction is applied to these averages  $(v_{dW}^{corrected}, s_{dW}^{corrected})$
        \item Lastly, the parameters are updated using the information from the calculated averages\\
        \[v_{dW}=\beta_{1}v_{dW}+(1-\beta_{1})\frac{\partial J}{\partial W}\]
        \[
        s_{dW}=\beta_{2}s_{dW}+(1-\beta_{2}) \left(\frac{\partial J}{\partial W}\right)^{2}
        \]
        \[
         v_{dW}^{corrected}=\frac{v_{dW}}{1-(\beta_{1})^{t}}
        \]
        \[
         s_{dW}^{corrected}=\frac{s_{dW}}{1-(\beta_{1})^{t}}
        \]
        \[
        W=W-\alpha\frac{v_{dW}^{corrected}}{\sqrt{s_{dW}^{corrected}} + \epsilon}
        \]
        where:\\
        $v_{dW}$-  the exponentially weighted average of past gradients\\
        $s_{dW}$- the exponentially weighted average of past squares of gradients\\
        $\beta_{1}$- hyperparameter to be tuned\\
        $\beta_{2}$- hyperparameter to be tuned\\
        $\frac{\partial J}{\partial W}$ - cost gradient with respect to current layer\\
        W- the weight matrix (parameter to be updated)\\
        $\alpha$ - the learning rate\\
        $\epsilon$ - very small value to avoid dividing by zero\\
    \end{itemize}
    \item \textbf{RMSProp Optimizer: } RMSProp optimizer is an adaptive learning rate optimization algorithm that helps to speed up convergence. It does this by keeping an exponentially weighted average of past gradient squares and dividing the learning rate by this average. This process results in a more efficient and faster convergence of the algorithm.
    \[
    s_{dW} = \beta s_{dW} + (1-\beta)\left(\frac{\partial J}{\partial W}\right)^{2}
    \]
    \[
    W = W-\alpha\frac{\frac{\partial J}{\partial W}}{\sqrt{s_{dW}^{corrected}}+ \epsilon}
    \]
    
    where: \\
    s - the exponentially weighted average of past squares of gradients\\
    $\frac{\partial J}{\partial W}$-  cost gradient with respect to current layer weight tensor\\
    W - weight tensor\\
    $\beta$- hyperparameter to be tuned\\
    $\alpha$- the learning rate\\
    $\epsilon$- very small value to avoid dividing by zero.
    \item \textbf{Stochastic Gradient Descent(SGD) Optimizer: } Stochastic Gradient Descent (SGD) is a widely used optimization algorithm for training neural networks and other machine learning models. It is a variant of the gradient descent optimization algorithm that processes each training example individually rather than using the entire dataset in each iteration. This property makes it well-suited for large datasets. Optionally, partition the dataset into mini-batches of a fixed size.
    \begin{itemize}
        \item For each mini-batch or individual example:
        \item Compute the gradient of the loss with respect to the model parameters.
        \item Update the model parameters in the opposite direction of the gradient to minimize the loss.
    \end{itemize}.\\
    The update rule for the model parameters $\theta$ is given by:
    \[
    \theta_{t} = \theta_{t-1}-\alpha\bigtriangledown J(\theta_{t-1},x_{i}, y_{i})
    \]\\
    where: \\
    $\alpha : $  is the learning rate, a hyperparameter that controls the size of the steps taken during optimization.\\
    $\bigtriangledown J(\theta_{t-1},x_{i}, y_{i}$ is the gradient of the loss function $\mathbf{J}$ with respect to the parameters $\theta$ for the example $x_{i}, y_{i}$
\end{itemize}



\subsubsection{Regularization}
Regularization techniques are widely used in machine learning to reduce overfitting and improve the generalization performance of models. These methods introduce constraints or penalties to the training process, which encourage the model to be simpler and more robust. By doing so, they help to prevent the model from fitting too closely to the training data, thereby improving its ability to generalize to new, unseen data. \cite{goodfellow2016deep}, \cite{ansari2020building}, \cite{bishop2006pattern}
\begin{itemize}
    \item \textbf{Data Augmentation: }
    Having more data is the surest way to get better consistent estimators (ML model), in contrast having a small dataset will lead to the well-known problem of overfitting.
    Data augmentation refers to the technique of artificially increasing the size of a dataset by applying various transformations to the existing data. This is often used in deep learning, particularly in computer vision tasks, to improve the performance of neural networks.
    There are various types of data augmentation techniques, including:
    \begin{enumerate}
        \item  Flipping: horizontally or vertically flipping an image.
        \item Rotation: Rotating an image by a certain degree is the process of rotating the image based on a specific angle.
        \item Zooming: zooming in or out of an image.
        \item Translation: shifting an image horizontally or vertically.
        \item Cropping: cropping a portion of an image.
        \item Adding noise: adding random noise to an image.
        
    \end{enumerate}
    
    These techniques can be used individually or in combination to generate new data samples. By increasing the size of the dataset, the neural network is exposed to more variations of the same data, which helps improve its ability to generalize and make accurate predictions on new, unseen data. \cite{shorten2019survey}\\
    \item \textbf{Dropout: } is a technique used to reduce overfitting in neural networks. It works by randomly ignoring selected neurons during training, which prevents complex co-adaptations on the training data. 

    During the training process, some of the neurons are randomly dropped out, meaning that their contribution to the activation of downstream neurons is temporarily removed on the forward pass. Additionally, any weight updates are not applied to the neuron on the backward pass. 
    
    Simply put, the process of ignoring some of the neurons occurs during a particular forward or backward pass. The probability of randomly selecting nodes to be dropped out can be easily adjusted (e.g. 0.1\%) each weight update cycle.
    
    It is important to note that Dropout is only used during the training of a model and is not used when evaluating the model. \\
    \item \textbf{Early Stopping: } Early stopping is an alternative technique that can be used to prevent overfitting. This approach involves using the validation error to decide when to stop training the neural network.

    The biggest challenge in training a neural network is determining how long to train the model. If you train the model too little, it will result in underfitting in both the train and test sets. On the other hand, if you train it too much, it will overfit the training set and perform poorly on the test sets.
    
    The key is to train the network long enough that it can learn the mapping from inputs to outputs, but not so long that it overfits the training data. One possible solution is to treat the number of training epochs as a hyperparameter and train the model multiple times with different values. Then, you can select the number of epochs that results in the best accuracy on the train or a holdout test dataset.
    
    However, this solution requires training and discarding multiple models, which can be time-consuming and resource-intensive. \cite{goodfellow2016deep}\\
    \item \textbf{Ensembling: } Ensemble methods are machine-learning techniques that combine multiple models into one predictive model. There are two primary approaches to ensembling: Bagging and Boosting.:
    \begin{itemize}
        \item \textbf{Bagging} Bagging, which stands for bootstrap aggregation, is a technique that reduces the variance of an estimate by averaging multiple estimates. It involves training a large number of "strong" learners in parallel. A strong learner is a model that is relatively unconstrained. Bagging then combines all the strong learners to obtain a smoother and more accurate prediction.
        \item  \textbf{Boosting}
        Boosting is a family of algorithms that convert weak learners into strong learners. It involves a sequence of models, with each one focusing on learning from the mistakes of the previous model. The final result is a combination of all the weak learners, which results in a single strong learner.
        
        Bagging utilizes complex base models to “smooth out” their predictions, while boosting uses simple base models to "boost" their combined complexity.
    \end{itemize}
    Bagging uses complex base models and tries to “smooth out” their predictions, while boosting uses simple base models and tries to “boost” their aggregate complexity.\\
    \item \textbf{Injecting Noise: } Noise is a phenomenon commonly introduced to the inputs as a technique for dataset augmentation. In situations where the dataset is small, the neural network may tend to memorize the training dataset instead of learning the general mapping from the inputs to the outputs. In other words, the model may only memorize specific input examples and their respective outputs. To solve this problem and enhance the structure of the mapping, one approach is to add random noise to the inputs.

    By adding noise, the network becomes less prone to memorizing the training samples since they keep changing all the time. This leads to smaller network weights and a more robust network, which in turn results in lower generalization error.
    
    It's important to note that noise is only added during the training phase. No noise is added when the model is evaluated or used to make predictions on new data. Additionally, random noise can be added to other parts of the network during training. Some examples include:
    
    \begin{enumerate}
        \item \textbf{Noise Injection on Weights} Adding noise to the weights of a model can be seen as a form of regularization that encourages the model to not be overly sensitive to small changes in the weights. By doing so, the model is able to avoid finding only local minima and instead find global minima surrounded by flat regions. This can result in a more robust and accurate model.
        \item \textbf{Noise Injection on Outputs}  It is common for real-world datasets to have errors in their output labels. In order to address this issue, one approach is to explicitly model the noise in the labels. An example of this is the technique of Noise Injection on Outputs, which can be achieved through a process known as label smoothing.
    \end{enumerate}
    \item \textbf{L1 Regularization: } A regression model that uses the L1 regularization technique is called Lasso Regression. The objective of L1 regularization is to push some of the weight coefficients to zero, effectively performing feature selection. This results in a sparse model where only a subset of the input features are used. \cite{kukavcka2017regularization}

    The formula for L1 regularization is:\\
    Mathematically:\\
    
    \[
    \text{Loss}=\text{Error}(Y,\hat{Y})
    \]
    Following formula calculates the error With L1 Regularization function
    \[
    \text{Loss}=\text{Error}(Y-\hat{Y}) + \lambda\sum_{1}^{n}|\omega_{i}|
    \]
    
    where:\\
    \[
    \hat{Y}=\omega_{1} x_{1}+\omega_{2} x_{2}+...+\omega_{n}x_{n}+b
    \]
    L1 Regularization (or a variant of this concept) is a model of choice when the number of features is high Since it provides sparse solutions.\\
    
    \item \textbf{L2 Regularization: } L2 regularization is a technique used in regression models, and when it is applied, the model is called Ridge Regression. The primary difference between L1 and L2 regularization techniques is that L2 regularization adds a penalty term to the loss function, which is the squared magnitude of the coefficient. \cite{kukavcka2017regularization}

    Mathematical formula for L2 Regularization.
    \[
    \text{Loss}=\text{Error}(Y,\hat{Y})
    \]
    
    \[
    \text{Loss}=\text{Error}(Y-\hat{Y}) + \lambda\sum_{1}^{n}\omega_{i}^{2}
    \]
    
\end{itemize}

\subsubsection{Types of Deep learning architectures}

Deep learning architectures are neural network models that are specifically designed to learn and make predictions from complex datasets like images, speech, and text. There are a wide variety of algorithms and architectures used in deep learning to accomplish this task.
\begin{figure}[H]
    \centering\includegraphics[width=1\textwidth]{DeepLearning_Arch.PNG}
    \caption{Deep learning architectures \cite{madhavan2017deep}}
\end{figure}
\begin{itemize}
    \item \textbf{Supervised deep learning: }
    Supervised learning refers to the problem space wherein the target to be predicted is clearly labeled within the data that is used for training.
    \begin{itemize}
        \item \textbf{Convolutional neural networks: } A CNN is a multilayer neural network used for image processing, inspired by the animal visual cortex. It was first created by Yann LeCun for recognizing handwritten characters. Early layers detect basic features, while subsequent layers combine these features to extract higher-level attributes of the input image. The LeNet CNN architecture performs feature extraction and classification through multiple layers, including convolutional and pooling layers, a fully connected multilayer perceptron, and an output layer. 
        \begin{figure}[H]
            \centering
            \includegraphics[width=0.8\linewidth]{tex/img/CNN.jpeg}
            \caption{Convolutional Neural Network \cite{Saily_Shah}}
            \label{fig:CNN}
        \end{figure}
        The network is trained through back-propagation. A Convolutional Neural Network (CNN) is a highly effective multilayer neural network that was inspired by the visual cortex of animals. It is primarily used for image processing applications. The first-ever CNN was created by Yann LeCun, which revolutionized the recognition of handwritten characters such as postal codes.
        
        With its deep network architecture, CNN can detect basic features, such as edges, through its initial layers and then combine these features to extract higher-level attributes of the input image.
        
        The LeNet CNN architecture, consisting of multiple layers that perform feature extraction and classification, is a prime example of CNN's effectiveness. It includes convolutional and pooling layers, a fully connected multilayer perceptron, and an output layer that identifies features of the image. The network is trained through back-propagation, making it an even more efficient tool. \cite{alzubaidi2021review} \cite{ansari2020building}
        
        \item  \textbf{The Gated Recurrent Unit (GRU) networks} In 2014, a simpler version of the LSTM network was introduced, known as the Gated Recurrent Unit (GRU). The GRU has two gates, an update gate and a reset gate, which replace the output gate present in the LSTM. The update gate determines how much of the previous cell contents should be kept, while the reset gate determines how to combine the new input with the previous cell contents. By setting the reset gate to 1 and the update gate to 0, a GRU can function as a standard RNN.\cite{madhavan2017deep}
        
        \begin{figure}[H]
            \centering\includegraphics[width=1\textwidth]{GRU_layers.PNG}
            \caption{GRU networks \cite{bfortuner_mlglossary}}
        \end{figure}
        The LSTM is more expressive and can lead to better results with more data, while the GRU is simpler, can be trained more quickly, and can be more efficient in its execution.
    \end{itemize}

     \item \textbf{Recurrent Neural Network (RNN): } is a neural network that contains a hidden state which captures historical information up to the current timestep. The hidden state of the current state uses the same definition as the previous timestep, which makes the computation recurrent, hence its name.

    The RNN is a foundational network architecture from which other deep learning architectures are built. The primary difference between a typical multilayer network and an RNN is that an RNN may have connections that feed back into prior layers (or the same layer) instead of completely feed-forward connections. This feedback allows RNNs to retain memory of past inputs and model problems in time.
    
    RNNs include a rich set of architectures, and one popular topology is called Long Short-Term Memory (LSTM). The key differentiator is feedback within the network, which can appear from a hidden layer, the output layer, or some combination of both.
    
    RNNs can be unfolded over time and trained using standard backpropagation or a variant called backpropagation through time (BPTT).  \cite{sherstinsky2020fundamentals}\\
    \begin{figure}[H]
        \centering\includegraphics[width=0.8\textwidth]{RNN.PNG}
        \caption{Recurrent Neural Network \cite{bfortuner_mlglossary}}
    \end{figure}
    \item \textbf{Gated Recurrent Unit (GRU) Layer: } GRU supports: \\
    \textbf{hidden gate} the gating of hidden state, \\
    \textbf{Reset gate} controls how much of the previous hidden state we might still want to remember.\\
    \textbf{Update gate} controls how much of current hidden state is just a copy of the previous state
    The structure and math are as follow:
    \begin{figure}[H]
        \centering\includegraphics[width=0.8\textwidth]{GRU_layers.PNG}
        \caption{Gated Recurrent Unit \cite{bfortuner_mlglossary}}
    \end{figure}
    \item \textbf{Long short-term memory (LSTM): } Long Short-Term Memory (LSTM) is a type of recurrent neural network (RNN) architecture. It is specifically designed to address the vanishing gradient problem, which can occur in traditional RNNs. LSTMs are particularly useful for handling tasks that involve sequences of data, such as speech recognition, natural language processing, time series analysis, and more.

    The most remarkable feature of LSTMs is their ability to capture and remember long-term dependencies in sequential data while avoiding the vanishing gradient problem. This is a significant advantage over traditional RNNs, which can struggle with the training of long sequences. \cite{alzubaidi2021review} \cite{madhavan2017deep}
    
    Here are the main components and features of an LSTM:
    
    \textbf{Cell State (Ct):}
    
    The cell state in LSTM serves as a long-term memory. It remains unchanged throughout the chain of the LSTM network, with only some minor linear interactions. It works like a conveyor belt that runs through the entire sequence, and information can be added or removed from it as needed.\\
    
    \textbf{Hidden State (ht):}\\
    
    The hidden state in LSTM is essentially a short-term memory that plays a crucial role in capturing and retaining short-term dependencies in the sequence. It is determined by both the current input and the previous hidden state at each time step.\\
    \textbf{Input Gate: } 
    The input gate is responsible for determining the amount of newly received information that should be added to the cell state. It utilizes a sigmoid activation function to determine which values should be updated (values close to 1) and which values should be ignored (values close to 0).\\
    \textbf{Forget Gate: }
    The forget gate is responsible for determining which information from the cell state should be disregarded. It takes into account the previous hidden state and the current input to decide which parts of the cell state are no longer important. It utilizes a sigmoid activation function to produce output values ranging from 0 to 1.\\
    \textbf{Cell State Update: }
   Cell state update is determined by two gates - the input gate and the forget gate. These gates work together to decide what new information should be added to the cell state and what information should be removed from it. The input gate determines the new information to be stored, while the forget gate determines which information should be discarded.\\
    \textbf{Output Gate:}
    
    The output gate plays a crucial role in deciding the next hidden state. It utilizes two activation functions - sigmoid and tanh. Sigmoid determines which parts of the cell state should be outputted, while tanh generates a vector of new candidate values that are added to the hidden state.
    \href{https://ml-cheatsheet.readthedocs.io/en/latest/layers.html#lstm}{Layers}
    \begin{figure}[H]
        \centering
        \includegraphics[width=0.8\linewidth]{tex/img/LSTM_layer.PNG}
        \caption{Long short-term memory \cite{yu2023popular}}
        \label{fig:LSTM_layer}
    \end{figure}
    \end{itemize}

\end{itemize}

\subsubsection{Section Unsupervised learning}

Unsupervised learning involves training models without target labels present within the data using unsupervised architectures.\cite{coates2011analysis}
\begin{itemize}
    \item \textbf{Self-organized maps: } Self-Organizing Maps (SOM) are a type of unsupervised neural network invented by Dr. Teuvo Kohonen in 1982, also known as Kohonen maps. Unlike traditional artificial neural networks, SOMs create clusters of input data by reducing input dimensionality. \cite{alzubaidi2021review}, \cite{madhavan2017deep}

    \begin{figure}[H]
        \centering
        \includegraphics[width=0.5\textwidth]{Self_organizingMap.PNG}
        \caption{Self-organized map (SOM)}
        \label{fig:Self-organized map (SOM)}
    \end{figure}
    \textbf{Example applications:} Dimensionality reduction, clustering high-dimensional inputs to 2-dimensional output, radiant grade result, and cluster visualization
    \item \textbf{Autoencoders: } Autoencoders are a variant of artificial neural networks (ANNs) that consist of three layers: the input layer, the hidden layer, and the output layer. First, the input layer is encoded into the hidden layer using an appropriate encoding function. The number of nodes in the hidden layer is much less than the number of nodes in the input layer. This hidden layer contains a compressed representation of the original input. Finally, the output layer aims to reconstruct the input layer by using a decoder function.
    
    \begin{figure}[H]
        \centering
        \includegraphics[width=0.7\linewidth]{tex/img/Autoencoders.PNG}
        \caption{Autoencoders \cite{madhavan2017deep}}
        \label{fig:Autoencoders}
    \end{figure}
   During the training phase, an error function is used to calculate the difference between the input and output layers, and the weights are adjusted accordingly to minimize the error. \\ 
    \textbf{Example applications:} Dimensionality reduction, data interpolation, and data compression/decompression.\\ 
    \item \textbf{Restricted Boltzmann Machines: } A Restricted Boltzmann Machine (RBM) is a type of neural network that consists of two layers: input and hidden layers. As depicted in figure \ref{fig:Restricted Boltzmann Machines}, every node in the hidden layer is connected to every node in the visible layer. Unlike traditional Boltzmann machines, where nodes in the input and hidden layers are interconnected, RBMs restrict the connections between nodes within a layer due to computational complexity.
    
    \begin{figure}[H]
        \centering
        \includegraphics[width=0.6\textwidth]{RestrictedBoltzmannMachines.PNG}
        \caption{Restricted Boltzmann Machines \cite{madhavan2017deep}}
        \label{fig:Restricted Boltzmann Machines}
    \end{figure}
    During the training phase of RBMs, a stochastic approach is used to calculate the probability distribution of the training set. At the beginning of the training, each neuron is activated at random. The model also contains hidden and visible biases. The hidden bias is used in the forward pass to build the activation, while the visible bias helps in reconstructing the input. RBMs are also called generative models because the reconstructed input is always different from the original input. Additionally, due to the built-in randomness, the same predictions result in different outputs. RBMs is a deterministic model, and that makes it significantly different from Autoencoder. \cite{madhavan2017deep} \cite{alzubaidi2021review} \\
    \textbf{Example applications:} Dimensionality reduction and collaborative filtering.\\
    \item \textbf{Deep belief networks: } Deep belief networks (DBN) are multilayer networks that typically have several hidden layers, making them deep. Each pair of connected layers in a DBN is a Restricted Boltzmann Machine (RBM). The input layer of the DBN represents the raw sensory inputs, while each hidden layer learns abstract representations of this input. The output layer is treated differently and is responsible for network classification during training. The DBN is trained in two steps: unsupervised pretraining and supervised fine-tuning. \cite{sohn2021deep}, \cite{coates2011analysis}
    
    \begin{figure}[H]
        \centering
        \includegraphics[width=0.7\textwidth]{DeepBelief_Networks.PNG}
        \caption{Deep belief networks architecture \cite{madhavan2017deep}}
        \label{fig:Deep belief networks}
    \end{figure}
    During unsupervised pretraining, each Restricted Boltzmann Machine (RBM) is trained to reconstruct its input. For instance, the first RBM reconstructs the input layer to the first hidden layer. Similarly, the next RBM is trained by using the outputs of the previous hidden layer as the inputs, and so on until each layer is pretrained. 

    After pretraining, fine-tuning commences, and the output nodes are assigned labels to give them meaning, i.e., to signify what they represent in the context of the network. The final step involves applying full network training using either gradient descent learning or back-propagation to complete the training process. \\
    \textbf{Applications: } object detection, information processing, natural language understanding processing, etc. \\
    \item \textbf{Deep stacking networks: } Deep stacking networks, also known as deep convex networks, are different from traditional deep learning frameworks. Although they consist of a deep network, they are actually a deep set of individual networks, each with its own hidden layers. 

    This architecture was developed to solve one of the main problems with deep learning: the complexity of training. Each layer in a deep learning architecture exponentially increases the complexity of training, but the DSN views training as a set of individual training problems, instead of a single problem.
    
    The DSN consists of a set of modules, each of which is a subnetwork in the overall hierarchy of the DSN. For example, in one instance of this architecture, three modules are created for the DSN. Each module consists of an input layer, a single hidden layer, and an output layer. Modules are stacked one on top of another, where the inputs of a module consist of the prior layer outputs and the original input vector. 
    
    This layering allows the overall network to learn more complex classifications than would be possible given a single module. \cite{deng2014tutorial}, \cite{madhavan2017deep}
    
    \begin{figure}[H]
        \centering
        \includegraphics[width=0.5\textwidth]{DeepStackNetworks.PNG}
        \caption{Deep stacking networks architecture \cite{madhavan2017deep}}
        \label{fig:Deep stacking networks}
    \end{figure}
    \textbf{Example applications:} Information retrieval and continuous speech recognition
    

\end{itemize}
 % Wygodnie jest trzymać każdy rozdział w osobnym pliku.
\section{SOTA of object detection}
\subsection{Introduction to Object Detection}
\href{https://www.mathworks.com/discovery/object-detection.html#:~:text=Object%20detection%20is%20a%20computer,learning%20to%20produce%20meaningful%20results.}{Object detection} is a computer vision technique for locating instances of objects in images or videos. Object detection algorithms typically leverage machine learning or deep learning to produce meaningful results. When humans look at images or videos, we can recognize and locate objects of interest within a matter of moments. The goal of object detection is to replicate this intelligence using a computer. \cite{zhao2019object}, \cite{ansari2020building}
\subsubsection{Why Object Detection}
Object detection is a fundamental task in computer vision that involves identifying and localizing objects within an image or a video. It plays a pivotal role in various real-world applications, contributing to advancements in technology and enhancing our daily lives. The importance of object detection lies in its ability to enable machines to interpret and understand visual information, allowing them to make informed decisions and interact intelligently with the environment.\cite{pathak2018application}, 
\begin{enumerate}
    \item \textbf{Automation and Efficiency:} Object detection is a cornerstone in the development of automated systems, empowering machines to perceive and respond to their surroundings. This is particularly crucial in industries such as manufacturing, where the automation of tasks, such as quality control and inventory management, leads to increased efficiency and reduced operational costs.
    \item \textbf{Enhanced Security and Surveillance:}  In the realm of security and surveillance, object detection is instrumental in identifying potential threats or anomalies. Surveillance cameras equipped with object detection algorithms can automatically detect and alert authorities to suspicious activities, enhancing public safety in crowded spaces, transportation hubs, and critical infrastructure.
    \item \textbf{Autonomous Vehicles:} The advent of autonomous vehicles relies heavily on object detection to interpret the dynamic environment. Cars equipped with object detection systems can identify pedestrians, vehicles, and obstacles, enabling safer navigation and reducing the likelihood of accidents.
    \item \textbf{Medical Imaging and Diagnosis:}  In the field of healthcare, object detection plays a vital role in medical imaging. It aids in the detection and localization of abnormalities in radiological images, facilitating early diagnosis and improving patient outcomes.
    \item \textbf{Retail and Customer Experience:} Object detection is utilized in retail for tasks such as inventory management, shelf monitoring, and cashierless checkout systems. These applications streamline operations and enhance the overall customer experience by reducing waiting times and optimizing stock levels.
\end{enumerate}

\textbf{Real-world Scenarios:}\\
\begin{enumerate}
    \item \textbf{Smart Cities:} object detection is integral to the development of smart cities. In urban environments, it can be used for traffic management, waste management, and monitoring public spaces, contributing to more efficient and sustainable urban living.
    \item \textbf{Search and Rescue Operations:} In disaster-stricken areas, object detection aids in search and rescue operations by identifying and locating individuals in need of assistance. Drones equipped with object detection capabilities can cover large areas quickly, improving the efficiency of rescue efforts.
    \item \textbf{Environmental Monitoring: } Object detection is applied in environmental science for monitoring wildlife, tracking deforestation, and studying biodiversity. It enables researchers to gather crucial data for conservation and ecological studies.
    \item \textbf{Augmented Reality:} Object detection is a key component in augmented reality applications, where virtual elements are seamlessly integrated with the real world. This technology enhances user experiences in gaming, education, and various interactive scenarios.
\end{enumerate} \cite{ansari2020building}, \cite{pathak2018application}

\subsection{Object detection stage}
Object detection can be classified into two stages according to the detection process steps:
\href{https://www.ijert.org/object-detection-using-yolo-and-mobilenet-ssd-a-comparative-study}{\textcolor{blue}{Source}}
\begin{itemize}
    \item \textbf{One-stage detector}  is a simple regression problem that takes input and learns probability classes and bounding box coordinates. YOLO, YOLO v2, SSD, RetinaNet, etc. fall under one-phase detectors. Object detection is an advanced form of imaging classification where a neural network predicts objects in an image and draws attention to them in the form of bounding boxes. \cite{oneStage}
    \item \textbf{Two-stage detector} Where detection is completed in two steps, the first step uses regional design networks to create areas of interest with a high probability of being objects. The second step is object detection, which performs the final classification and regression of the bounding box of the objects. RCNN, Fast RCNN, SPPNET, Faster RCNN, etc., are some of the two-stage detectors. \cite{du2020overview}
\end{itemize}

\begin{figure}[H]
    \centering
    \includegraphics[width=0.7\textwidth]{ObjectDetection_Stage.PNG}
    \caption{Stage  or classification of Object Detection}
    \label{fig:Stage of Object Detection}
\end{figure}

\subsection{Two-Stage/Proposal: }
  Two-stage object detection algorithms typically follow a two-step process: region proposal and object classification or refinement. One of the most popular two-stage object detection frameworks is the region-based Convolutional Neural Network (R-CNN) and its variants. \cite{du2020overview}, \cite{ansari2020building}
\subsubsection{Region-Based Convolutional Neural Network: } \underline{\textcolor{blue}{\href{https://arxiv.org/pdf/1311.2524.pdf}{R-CNN}}}
 A Region-Based Convolutional Neural Network (R-CNN) is a type of convolutional neural network designed for object detection in images.R-CNN predominantly uses region proposals using a selective search (SS) approach to pre-compute the priors. The features between these regions are not shared, so we have to extract features individually for each region-of-interest (RoI) generated using an SS approach. In contrast, R-CNNs are designed to not only classify objects but also to locate and delineate their boundaries within the image.
The R-CNN architecture was introduced by Ross Girshick, Jeff Donahue, Trevor Darrell, and Jitendra Malik in their 2014 paper titled "Rich Feature Hierarchies for Accurate Object Detection and Semantic Segmentation." The architecture has since evolved into several versions, including Fast R-CNN, Faster R-CNN, and Mask R-CNN.
        
    \begin{figure}[H]
         \centering
         \includegraphics[width=0.6\textwidth]{R_CNN.PNG}
         \caption{R-CNN Model(Source:Girshick et al.)}
            \label{fig:R-CNN Model}
    \end{figure}
\textbf{R-CNN comprises of the following three modules:}\\
        \begin{enumerate}
            \item \textbf{Region proposal: } The R-CNN algorithm starts by identifying regions in an image that may contain objects. These regions are known as region proposals. They are referred to as proposals because they may or may not contain objects, and the goal of the learning function is to remove areas that do not contain objects. These region proposals are essentially bounding boxes around the objects.
            \item \textbf{Feature extraction:} To identify objects in an image, the first step is to crop out the region proposals and resize them. These resized images are then sent through a standard CNN for feature extraction. The original research paper employed AlexNet for this purpose. The CNN extracts 4,096-dimensional feature vectors from each region.
            \item  \textbf{Classifier: } The extracted features are classified by using the standard classification algorithms, such as the linear SVM model 
        \end{enumerate}
R-CNN was the first successful deep learning–based object detection system, but it suffered a serious issue with respect to performance. Its time performance problem is because of the following:
    \begin{itemize}
         \item For feature extraction, each region proposal undergoes approximately 2,000 passes per image in the CNN.
        \item Three models are trained: CNN for feature extraction, classifier for image class prediction, and regression for bounding box refinement. Training is compute-intensive, increasing computation time.
        \item Due to the large number of regions, CNN predictions are slow for each proposal.
    \end{itemize}
        
\subsubsection{Fast R-CNN:} To overcome the limitations of R-CNNs, Ross Girshick from Microsoft published a paper in 2015 titled “Fast R-CNN” that proposed a single model to learn and output regions and classifications directly \href{https://arxiv.org/pdf/1504.08083.pdf}{\textcolor{blue}{Fast R-CNN:}}
\href{https://medium.com/alegion/deep-learning-for-object-detection-and-localization-using-fast-r-cnn-85d52e3928a1}{\textcolor{blue}{[2]}}\\
On the contrary, Fast R-CNN trains a deep VGG-16 network, 9x faster than R-CNN and 213x faster at test time, achieving a higher mAP on PASCAL VOC 2012. How does it achieve this enormous gain in speed? Fast R-CNN evaluates the network and extracts features for the whole image once, instead of extracting features from each RoI, which are cropped and rescaled every time in R-CNN. It then uses the concept of RoI pooling, a special case of pyramid pooling used in SPPNet, to give a feature vector of the desired length. This feature vector is then used for classification and localization. This method is more effective than R-CNN because the computations for overlapping regions are shared.
        

\textbf{Building blocks of Fast R-CNN:}
    \begin{enumerate}
        \item Region proposal network
        \item Feature extraction using CNN
        \item RoI pooling layer: This is where the real magic of Fast R-CNN happens
        \item Classification and Localization
    \end{enumerate}
        
    \begin{figure}[H]
        \centering
        \includegraphics[width=1\textwidth]{Fast_RCNN_ARch.PNG}
        \caption{Architectural Design of Fast R-CNN }
        \label{fig: Fast R-CNN}
    \end{figure}
\href{https://www.mathworks.com/help/vision/ug/getting-started-with-r-cnn-fast-r-cnn-and-faster-r-cnn.html#d117e11449}{\textcolor{blue}{Image Source}}
        
The region proposal network and feature extraction modules are very similar to what we have seen in R-CNN. But instead of passing each cropped and re-scaled RoI, the entire input image is passed through a feature extractor like VGG-16 to produce a convolutional feature map. The features (i.e., the convolutional feature map) are combined with the region proposal network, which uses an SS approach, to form a fixed-length feature vector in the RoI pooling layer. Each of these feature vectors is then passed along to the classification and localization modules. The classification module classifies K+1 (1 for background) object classes using a softmax probability. The localization module outputs four real-valued numbers for each K object class. \cite{girshick2015fast}
\subsubsection{Faster R-CNN}
Faster R-CNN is an improved version of Fast R-CNN from the training speed and detection accuracy perspectives.
Faster R-CNN added what they called a Region Proposal Network (RPN) in an attempt to get rid of the selective search algorithm and make the model completely trainable end-to-end.

\begin{figure}[H]
    \centering
    \includegraphics[width=0.7\textwidth]{Faster_RNN.PNG}
    \caption{Architectural Design of Faster R-CNN}
    \label{fig:Faster R-CNN}
\end{figure}

\begin{enumerate}
    \item \textbf{Region Proposal Network: } An RPN is a fully convolutional neural network that predicts object bounds and objectness scores simultaneously at each position of the image.
    An RPN is a deep CNN that takes an image input and generates the output as a set of rectangular object proposals. Each rectangular proposal has an “objectness” score.
    
    \begin{figure}[H]
        \centering
        \includegraphics[width=1\textwidth]{RPN.PNG}
        \caption{Region Proposal Networks (RPN) [image source: Shaoqing Ren, et al.]}
        \label{fig:RPN}
    \end{figure}
    RPN has a classifier and a regressor. The authors have introduced the concept of anchors. The anchor is the central point of the sliding window.
    The classifier determines the probability of a proposal having the target object. Regression
    regresses the coordinates of the proposals.
    Multiple region proposals are predicted at each sliding window location. Assuming the maximum number of proposals at each window location is k, the total number of bounding box coordinates will be 4k, and the number of object classes will be 2k (one for the probability of being an object and the other for the probability of not being an object). These region boxes at each window are called anchors.\\
    \item \textbf{Fast R-CNN: } The second part of the faster R-CNN is the detection network. This part is exactly the same as the Fast R-CNN (as described earlier). The Fast R-CNN takes input from the RPN to detect objects in images. 
\end{enumerate} \cite{ansari2020building}, \cite{salvador2016faster}


\subsubsection{Mask R-CNN}
\textcolor{red}{\href{https://arxiv.org/pdf/1703.06870.pdf}{\textcolor{blue}{Mask R-CNN}}}
The Mask R-CNN extends the Faster R-CNN. The Mask R-CNN adds an extra branch for predicting an object mask along with the object class and bounding box coordinates.
Here is how the Mask R-CNN differs from its predecessor, the Faster R-CNN:
\begin{itemize}
    \item The Faster R-CNN has two outputs: a class label and bounding box coordinates.
    \item The Mask R-CNN has three outputs: a class label, bounding box coordinates, and an object mask.
\end{itemize}
In the Mask R-CNN, each pixel is classified into a fixed set of categories without differentiating object instances. It introduces a concept called pixel-to-pixel alignment between the output and input layers of the neural network. The class of each pixel determines the masks in the ROI.

\begin{figure}[H]
    \centering
    \includegraphics[width=1\textwidth]{MaskR-CNN.PNG}
    \caption{Mask R-CNN network architecture}
    \label{fig:MaskR-CNN}
\end{figure}
As shown in Figure \ref{fig:MaskR-CNN}, the network consists of three components \textbf{modules—backbone, RPN, and output head}.
\begin{itemize}
    \item \textbf{Backbone} The backbone networks are commonly seen in object detection model architectures. The original paper describes using ResNet-50 and ResNet-101[ \textcolor{red}{\href{https://arxiv.org/pdf/1612.03144.pdf}{Ref1}} ]
    The backbone’s main role is feature extraction.
    In addition to ResNet, a feature pyramid network (FPN) is used to extract the finer feature details of the image.
    \begin{figure}[H]
        \centering\includegraphics[width=1\textwidth]{BackBone_FPN.PNG}
        \caption{BackBone and FPN Tuning architecture \protect\href{https://medium.com/@freshtechyy/fusing-backbone-features-using-feature-pyramid-network-fpn-c652aa6a264b}{[\textcolor{blue}{Ref}} ]}
    \end{figure}
The FPN consists of decreasing-size layers of a CNN, in which case each forward layer has fewer neurons.

        \item \textbf{Feature Pyramid Network (FPN): } is a neck network that combines features of different resolutions obtained from a backbone network, such as ResNet. A CNN-based backbone applies convolution layers to an input image, which results in a set of feature maps with decreasing resolution due to pooling or convolution with a stride different from one.\\
        The FPN includes a bottom-up pathway and a top-down pathway. In the bottom-up pathway, a backbone network, like ResNet, is used to perform feature extraction to extract features with decreasing levels of spatial resolution.
        As the levels of resolution decrease, the semantic meaning of the feature maps increases, as indicated by the thickness of box boundaries in blue.\\
        In the top-down pathway, the feature maps are fused to have both rich semantic meaning and accurate spatial information as shown in the figure \ref{fig:FPN_Arch} below.
        \begin{figure}[H]
            \centering
            \includegraphics[width=0.7\textwidth]{FPN_Top_bott.PNG}
            \caption{Feature Pyramid Network (FPN)}
            \label{fig:FPN_Arch}
        \end{figure}
         \item \textbf{Output Head: } The last module consists of the Faster R-CNN with an additional output branch. \cite{he2017mask}
\end{itemize}
   
\subsection{One stage detector: } 
They can process images in real-time and are suitable for applications where low latency is crucial, such as real-time object detection in video streams or robotics.

\subsubsection{Single-Shot Multibox Detection: }

SSD is designed for object detection in real time. An R-CNN and its variants are detectors that work in two stages. They have two specialized networks: one network creates the region proposals to predict bounding boxes, and the other network predicts the object classes. These detectors are fairly accurate, but they come with a high computational cost. As a result, they are not suitable for detecting objects in real-time streaming videos.
A single-shot object detector predicts both the bounding boxes and the object classes in a single forward pass of the network.\cite{liu2016ssd} \href{https://arxiv.org/pdf/1512.02325.pdf}{\textcolor{blue}{[Original Paper]}}
\begin{itemize}
    \item \textbf{SSD Network Architecture: } The SSD approach is a method that uses a feed-forward convolutional network to generate a set of fixed-size bounding boxes and scores, which helps detect the presence of object class instances in those boxes. In the final stage, a non-maximum suppression step is performed to obtain the ultimate detections. \\
    \begin{itemize}
        \item \textbf{Grid cell: } Instead of using sliding window, SSD divides the image using a grid and has each grid cell be responsible for detecting objects in that region of the image. Detection of objects simply means predicting the class and location of an object within that region. If no object is present, we consider it the background class, and the location is ignored. 
        SSD only needs an input image and ground truth boxes for each object during training.
        In a convolutional fashion, we evaluate a small set
        of default boxes of different aspect ratios at each location in several feature maps with
        different scales (e.g. 8x8 and 4x4 in (b) and (c)). For each default box, we predict
        both the shape offsets and the confidences for all object categories ((c1; c2; ...; cp)).
        At training time, we first match these default boxes to the ground truth boxes.
        \begin{figure}[H]
            \centering
            \includegraphics[width=\textwidth]{SSD_GridCell.PNG}
            \caption{SSD:Single Shot Multi Box Detector \textcolor{red}{\protect\href{https://arxiv.org/pdf/1512.02325.pdf}{image source}}}
            \label{fig:SSD_grid}
        \end{figure}
        
        \item \textbf{Anchor Boxes:} In object detection, we are seeking to identify and localize objects as they appear in an image. Object detection differs from image classification because there may be multiple objects of the same or different classes present in the image, and object detection seeks to accurately predict all of these objects.
        Object detection models tackle this task by breaking the prediction step into two pieces: 
        \textcolor{red}{\href{https://blog.roboflow.com/what-is-an-anchor-box/}{Ref: Anchor-box}}
        \begin{enumerate}
            \item First, they predict a bounding box through regression and
            \item Second, by predicting a class label through classification.
        \end{enumerate}
        
        \begin{figure}[H]
            \centering
            \includegraphics[width=0.2\textwidth]{Anchor.PNG}
            \caption{Anchor}
            \label{fig:anchor}
        \end{figure}
        To predict and localize many different objects in an image, most state-of-the-art object detection models such as SSD, EfficientDet, and the YOLO models start with anchor boxes as a prior, and adjust from there. These anchor boxes are pre-defined, and each one is responsible for a size and shape within a grid cell.
        
        Anchors are one or more rectangular shapes set at each convolution point of the feature map. 
        Each grid cell in SSD can be assigned multiple anchor/prior boxes.
        
        
        In Figure \ref{fig: anchor box}, there are five rectangular anchors (shown in red outlines) set at a point (shown in blue).
        \begin{figure}[H]
            \centering
            \includegraphics[width=0.5\textwidth]{Anchor_Boxes.PNG}
            \caption{Anchor Boxes}
            \label{fig: anchor box}
        \end{figure}
        In SSD, typically five anchor boxes are selected at each point. Each of these anchors acts as a detector.
        The varying size of these detectors allows them to detect objects of different sizes. Smaller detectors will detect smaller objects, and larger detectors are capable of detecting larger objects.\\
        \item \textbf{Default Boxes and Aspect Ratios: }  doesn’t use k-means to find the anchors. Instead it uses a mathematical formula to compute the anchor sizes. Therefore, SSD’s anchors are independent of the dataset (by the way, the SSD paper calls them “default boxes”)\textcolor{red}{\href{https://arxiv.org/pdf/1512.02325.pdf}{Ref:Original Paper}}.  \href{https://machinethink.net/blog/object-detection/}{SSD}

        It is important to note that these anchors are chosen beforehand as constants. In SSD, a set of fixed “default anchors” is mapped at each convolution point.\\
        They have associated a set of default bounding boxes with each feature map cell, for multiple feature maps at the top of the network. The default boxes tile the feature map in a convolutional manner so that the position of each box relative to its corresponding cell is fixed.\\
            \end{itemize}
    \item \textbf{Model Architecture: } An SSD neural network consists of two components: \textbf{base network and prediction network.}
    \begin{enumerate}
        \item \textbf{Base network:} The base network is a deep convolutional network that is truncated before any classification layer. For example, remove the fully connected layer of ResNet or VGG to create the base network for SSD. The base network is used for feature extraction from the input images.
        \item \textbf{Detection network:} To the base network, attach some extra convolutional layers that will actually do the prediction of bounding boxes and object classes. 
        \begin{figure}[H]
        \centering
        \includegraphics[width=\textwidth]{SSD_ARch.PNG}
        \caption{single shot detection models: SSD \cite{liu2016ssd}}
        \label{fig: single shot detection models: SSD}
        \end{figure}
    The detection network has the following characteristics.
        \begin{itemize}
            \item These layers decrease in size progressively and allow
            predictions of detections at multiple scales
            \item  The convolutional model for predicting detections is different for each feature layer
            \item Each added feature layer (or optionally an existing
            feature layer from the base network) can produce a fixed set of detection predictions
            using a set of convolutional filters.
        \end{itemize}
    \end{enumerate}
    \item \textbf{Matching System: } During training, we need to select default boxes that correspond to ground truth detection and train the network accordingly. For each ground truth box, we choose from default boxes that vary in location, aspect ratio, and scale.  The SSD uses IoU (intersection over union) to match the default boxes with the ground truth. This is done by determining the overlap between them. The IoU-based overlap is also known as the Jaccard overlap. If the overlap between the default box and the ground truth is 0.5 or more, it is considered a match. This process of matching is repeated at each layer, which allows the network to learn at scale. Initially, the SSD uses the default boxes as predictions and then tries to regress and come closer to the ground truth bounding boxes. \cite{ansari2020building}, \cite{liu2016ssd}

\end{itemize}

\subsection{Object Detection Evaluation Metrics}
Object detection is a computer vision task where the goal is to identify and locate objects of interest within an image or a video. Several evaluation metrics are commonly used to assess the performance of object detection algorithms. Here are some of the key metrics:
\subsubsection{Intersection over Union (IoU)}
Intersection over union (IoU), also known as the Jaccard index, is one of the most commonly used evaluation metrics in object detection algorithms. It is used to evaluate the performance of object detection by comparing the ground truth bounding box to the predicted bounding box and IoU.
In object detection, we create training sets by drawing bounding boxes around objects for labeling. These bounding boxes in the training set are also known as the ground truth. During the model learning, the object detection algorithm predicts bounding boxes and compares them against the ground truth. IoU is used to evaluate how closely the predicted bounding box overlaps with the ground truth. \cite{ansari2020building}, \cite{bfortuner_mlglossary}, \cite{liu2016ssd}

\begin{figure}[H]
    \centering
    \includegraphics[width=0.6\textwidth]{Intersection_overUnion_(IoU).PNG}
    \caption{The predicted bounding box and Ground-truth bounding box}
    \label{fig:PBGTB}
\end{figure}

\textbf{Computing Intersection over Union can therefore be determined as:}\\

\begin{figure}[H]
    \centering
    \includegraphics[width=0.6\textwidth]{Intersection_OverUnion.PNG}
    \caption{Intersection OverUnion}
    \label{fig:Intersection_OverUnion}
\end{figure}

\subsubsection{Mean Average Precision (mAP)}

Mean Average Precision (mAP) is commonly used to analyze the performance of object detection and segmentation systems. Many object detection algorithms, such as Faster R-CNN, MobileNet SSD, and YOLO, use mAP to evaluate their models. The mAP is also used across several benchmark challenges, such as Pascal, VOC, COCO, and more. The mean of average precision(AP) values are calculated over recall values from 0 to 1. \cite{padilla2020survey}, \cite{ansari2020building},  \\
mAP formula is based on the following sub metrics:
\begin{itemize}
    \item Confusion Matrix
    \item Intersection over Union(IoU)
    \item Recall,
    \item Precision
\end{itemize}
\begin{enumerate}
    \item \textbf{Confusion Matrix: }A confusion matrix is a table that is often used to evaluate the performance of a classification algorithm on a set of labeled data for which the true values are known. It provides a summary of the classification results, breaking down the predicted and actual classes into four categories: true positives (TP), true negatives (TN), false positives (FP), and false negatives (FN). \cite{padilla2020survey}, \cite{li2019analysis}
    \begin{itemize}
        \item \textbf{True Positives (TP): } The model predicted a label and matches correctly as per ground truth.\\
        \item \textbf{True Negatives (TN): } The model does not predict the label and is not a part of the ground truth.\\
        \item  \textbf{False Positives (FP): } The model predicted a label, but it is not a part of the ground truth (Type I Error).\\
        \item \textbf{False Negatives (FN): } The model does not predict a label, but it is part of the ground truth. (Type II Error).
        \begin{figure}[H]
            \centering
            \includegraphics[width=0.7\textwidth]{confusion_matrix.PNG}
            \caption{Confusion matrix table}
            \label{fig:Confusion_Matrix}
        \end{figure}
    \end{itemize}
    \item \textbf{Precision: } Precision measures how well you can find true positives(TP) out of all positive predictions. (TP+FP).
    $$Precision=\frac{TP}{TPTP+FP}$$

    \item \textbf{Recall: } Recall measures how well you can find true positives(TP) out of all predictions(TP+FN).
    $$Recall=\frac{TP}{TPTP+FN}$$
    
  
\end{enumerate}


\subsection{YOLO}
YOLO is a real-time object detection algorithm that is designed to be fast and accurate. It uses a single convolutional neural network to simultaneously predict the bounding boxes and class probabilities of objects in an image. What sets YOLO apart from other object detection algorithms is that it trains on the full image and is set up to solve regression problems. This means that it does not require a complex processing pipeline, which makes it extremely fast.YOLO is a system that simplifies object detection by treating it as a single regression problem. It predicts bounding box coordinates and class probabilities directly from image pixels. With YOLO, you can detect objects in an image with just one look and get information about what objects are present and where they are located. \cite{redmon2016you}, \cite{ansari2020building}

\subsubsection{Detection Algorithm}
The YOLO design enables end-to-end training and real-time speeds while maintaining high average precision. \\
It divides the image into an SxS grid and for each
grid cell predicts \textbf{B} bounding boxes, confidence for those boxes,
and C-class probabilities. These predictions are encoded as a tensor.
\[
S x S x (B*5 + C)
\]
\\
\begin{figure}[H]
    \centering
    \includegraphics[width=0.8\textwidth]{YOLO_Detection.PNG}
    \caption{YOLO Model}
    \label{fig:YoloV1_Model}
\end{figure}

Each bounding box consists of 5 predictions: x, y, w, h,
and confidence. confidence scores reflect how confident the model is.
\[
Confidence =Pr(Object)*IOU_{pred}^{truth}
\]
\subsubsection{YOLO Architectural Design}
The YOLO network architecture takes inspiration from the GoogLeNet model that is used for image classification. It consists of 24 convolutional layers followed by 2 fully connected layers. In contrast to the inception modules used in GoogLeNet, the YOLO network employs 1x1 reduction layers that are then followed by 3x3 convolutional layers. The full network is shown in Figure[\ref{fig:YOLO_arc}].
\begin{figure}[H]
    \centering
    \includegraphics[width=\textwidth]{Yolo_Arch.PNG}
    \caption{YOLO Model}
    \label{fig:YOLO_arc}
\end{figure}
The convolutional layers were pre-trained on the ImageNet 1000-class competition dataset. For pretraining, YOLO uses the first 20 convolutional layers from Figure \ref{fig:YOLO_arc}, followed by an average-pooling layer and a fully connected layer. The final layer predicts both class probabilities and bounding box coordinates. A linear activation function is used for the final layer and
all other layers use the following leaky rectified linear activation:
 \[
 \phi(x)=\begin{cases}
           x, & \text{if } x>0 \\
          0.1x, & \text{otherwise}
      \end{cases}
\]

\subsubsection{Limitations of YOLO}
\begin{itemize}
    \item It struggles with small objects that come in groups, such as flocks of birds.
    \item It can predict only one class of objects within a cell grid.
    \item It does not predict well if the object has an unusual aspect ratio that was not seen in the training set.
    \item Its accuracy is less than some of the state-of-the-art algorithms, such as the Faster R-CNN
\end{itemize}

\subsection{Transfer Learning and Pre-trained Models:}

Transfer learning is a technique in deep learning where a pre-trained model for a particular task is used as a starting point for another model that performs a similar task. This approach provides a faster and easier way to update and retrain a network as compared to training a network from scratch. It is commonly used in various applications, such as object detection, image recognition, and speech recognition. \cite{you2021logme}
\href{https://www.mathworks.com/discovery/transfer-learning.html?s_tid=srchtitle_support_results_1_Transfer%2520Learning}{\textcolor{blue}{Source Matlab}}

Transfer learning is a popular technique because:
\begin{itemize}
    \item It enables you to train models with less labeled data by reusing popular models that have already been trained on large datasets.
    \item  It can reduce training time and computing resources. With transfer learning, the weights are not learned from scratch because the pre-trained model has already learned the weights based on previous learning.
    \item You can take advantage of model architectures developed by the deep learning research community, including popular architectures such as GoogLeNet and ResNet.
\end{itemize}
\begin{figure}[H]
    \centering\includegraphics[width=1\linewidth]{Transfer_Learning_Model.PNG}
    \caption{Transfer Learning \protect\href{https://www.mathworks.com/help/deeplearning/ug/pretrained-convolutional-neural-networks.html}{\textcolor{blue}{Image source}}}
    \label{fig:Transfer Learning}
\end{figure}
\subsubsection{Pre-trained Model}
There are several pre-trained models available for object detection. Below are some of the most popular ones:\\
Alexnet, googlenet(ImageNet), goolgenet(Places365), resnet18, resnet50, resnet101, vgg16, vgg19, inceptionv3, inceptionresnetv2, squeezenet, densenet201, mobilenetv2, shufflenet, xception, nasnetmobile, nasnetlarge. \\

When choosing a neural network, it is important to consider its accuracy, speed, and size, as these are the most crucial characteristics. Typically, there is a tradeoff between these factors. To make an informed decision, you can refer to the plot below, which compares the ImageNet validation accuracy with the time taken to make a prediction using the neural network. \cite{li2019analysis}
\begin{figure}[H]
    \centering
    \includegraphics[width=1\textwidth]{Pre-trained_comparision.PNG}
    \caption{Compare Pretrained Neural Networks}
    \label{fig: Pretrained Model Comparison}
\end{figure}

    % Umożliwia to również łatwą migrację do nowej wersji szablonu:
\section{Implemented object detection algorithm and Evolution of YOLO model} \vspace{-2em}
\subsection{YOLO}
YOLO (You Only Look Once) is an object detection system used for real-time object detection. It was introduced in 2016 by Joseph Redmon, Santosh Divvala, Ross Girshick, and Ali Farhadi. YOLO is known for its high speed and accuracy. The system is designed to classify and detect multiple objects in an image using a single forward pass of the neural network.
Traditional object detection systems use a sliding window approach where they apply a classifier at each location and scale of the image. This process is computationally expensive and time-consuming. On the other hand, YOLO divides the input image into a grid of cells and each cell is responsible for detecting an object. This approach reduces the number of bounding boxes that need to be processed and allows for real-time detection, this make it suitable for a wide range of applications such as self-driving cars, surveillance systems, and robotics. Additionally, YOLO can detect objects at different scales and aspect ratios, making it robust to variations in object size and orientation. \cite{terven2023comprehensive} \cite{redmon2016you} \href{https://deci.ai/blog/history-yolo-object-detection-models-from-yolov1-yolov8/}{\textcolor{blue}{Ref 1}}\\
2. \href{https://deci.ai/blog/yolov8-vs-yolo-nas-showdown-exploring-advanced-object-detection/}{\textcolor{blue}{[Ref]}}
3. \href{https://arxiv.org/pdf/2304.00501.pdf}{\textcolor{blue}{Ref: 3}}

\subsection{The Evolution of YOLO}
\begin{figure}[H]
    \centering
    \includegraphics[width=\textwidth]{evolution_of_YOLO.PNG}
    \caption{The evolution  of yolo [Source: \href{https://deci.ai/resources/webinar-open-source-llms-vs-apis/}{\textcolor{blue}{Deci. ai webinar}}]}
    \label{fig:YOLO-Evolution}
\end{figure}
\begin{enumerate}
    \item \textbf{YOLO: } YOLO introduced in 2016 for real-time object detection.
    Achieved a groundbreaking mAP of 63.4\% on PASCAL VOC2007 dataset.
    YOLO's efficiency in one network pass.
    Localization error limitations due to object count, aspect ratios, and down-sampling. \cite{redmon2016you}\\
    \item \textbf{YOLOv2 Improvements: } YOLOv2 introduced in 2017 with 9000+ categories.
    Improved with batch normalization, high-resolution classifier, and anchor boxes.
    Joint training for classification and detection.\cite{sang2018improved}\\
    \item \textbf{YOLOv3 Advancements: } YOLOv3 (2018) achieved 60.6\% mAP on MS COCO, 2x faster than previous versions.
    Introduced logistic regression for objectness scores and anchor box priors. \cite{redmon2018yolov3}\\
    \item \textbf{YOLOv4 Global Impact: } YOLOv4 (2020) maintained YOLO's philosophy with bag-of-freebies and bag-of-specials. Enhanced accuracy with image adjustments like mosaic augmentation and DropBlock. \cite{gai2023detection}
    \item \textbf{YOLOv5 Speed and Efficiency: } YOLOv5 (2020) by Ultralytics in PyTorch achieved 50.7\% AP with user-friendly features. \cite{wu2021real}\\
    \item \textbf{YOLOX Anchor-Free Innovation: } YOLOX (July 2021) exceeded previous YOLO versions, anchor-free with center sampling. Employed MixUP, Mosaic augmentations for improved accuracy. \cite{ge2021yolox}\\
    \item \textbf{YOLOR Multi-Task Learning: } YOLOR (May 2021) focused on a unified network for multiple tasks.
    Applied multi-task learning for classification, detection, and pose estimation.\\

    \item \textbf{PP-YOLOv2: }  Upgrades include ResNet101, PAN, Mish Activation, increased input size, and modifications to IoU-aware branch. \cite{huang2021pp}\\
    \item \textbf{Scaled-YOLOv4 Flexibility: } Scaled-YOLOv4 (CVPR 2021) introduced scaling-up and scaling-down techniques. YOLOv4-tiny for low-end GPUs, YOLOv4-large for cloud GPUs. \cite{wang2021scaled}\\ 
    \item \textbf{PP-YOLO: } A YOLOv3-based model by Baidu, leveraging PaddlePaddle, introducing ten tricks for accuracy without sacrificing speed.\cite{long2020pp}\\
    \item \textbf{PP-YOLOE: } Evolved from PP-YOLOv2 with anchor-free architecture, new backbone, TAL, ET-head, and VFL/DFL. \cite{xu2022pp}\\
    \item \textbf{YOLOv6 Industrial Framework: } 
    YOLOv6 (September 2022) for industrial applications with anchor-free detection.
    YOLOv6-L achieved 52.5\% AP and 70\% AP50 at 50 FPS. \cite{li2022yolov6}, \cite{terven2023comprehensive}
    \item \textbf{YOLOv7 Speed and Accuracy: } YOLOv7 (July 2022) surpassed others in speed and accuracy.
    E-ELAN, model scaling, and "bag-of-freebies" approach for efficiency. \cite{wang2023yolov7}, \cite{terven2023comprehensive}\\
    \item \textbf{DAMO-YOLO Real-Time Improvements: } DAMO-YOLO (November 2022) by Alibaba for real-time object detection. 
    Introduced MAE-NAS, Efficient-RepGFPN, ZeroHead, and knowledge distillation. \cite{xu2022damo}\\
    \item \textbf{YOLOv8 :}
    YOLOv8 (January 2023) by Ultralytics with anchor-free prediction and faster NMS. Mosaic augmentation during training for improved accuracy.
    YOLOv8 offers five scaled versions: YOLOv8n (nano), YOLOv8s (small), YOLOv8m (medium), YOLOv8l (large), and YOLOv8x (extra-large). YOLOv8x outperformed YOLOv5 on the MS COCO dataset with an impressive 53.9\% AP at 640 pixels and fast speed.\cite{yolov8}, \cite{terven2023comprehensive}
\end{enumerate}
\href{https://www.mdpi.com/2504-4990/5/4/83}{\textcolor{blue}{Source}}
\subsection{YOLO-NAS}
YOLO-NAS is a cutting-edge object detection model created using Deci's Neural Architecture Search technology, AutoNAC™. It offers unmatched real-time object detection capabilities and production-ready performance, surpassing other models such as YOLOv5, YOLOv6, YOLOv7, and YOLOv8.

The YOLO-NAS model undergoes a multi-phase training process, which involves pre-training on Objects365, COCO Pseudo-Labeled data, Knowledge Distillation (KD), and Distribution Focal Loss (DFL).

During the pre-training phase, the model is trained on Objects365, a comprehensive dataset that consists of 2 million images and 365 categories. Depending on the model variant, this process can take 25-40 epochs, as each epoch requires 50-80 minutes on 8 NVIDIA RTX GPUs. 

In addition, the COCO dataset provides an extra 123k unlabeled images, which are used to generate pseudo-labeled data. An accurate model is trained on COCO to label these images, which are then used to train our model with the original 118k train images.

Moreover, the YOLO-NAS architecture uses Knowledge Distillation (KD) and Distribution Focal Loss (DFL) techniques to enhance its training process.
YOLO-NAS training procedure is comprehensive and leverages multiple datasets, labelled and unlabelled, and supervised and unsupervised training procedures. \cite{YOLO-NAS}, \cite{terven2023comprehensive}
\begin{figure}[H]
    \centering
    \includegraphics[width=0.8\textwidth]{YOLO-NAS-Training.PNG}
    \caption{An overview of the YOLO-NAS training process \cite{richmond_yolo-nas} \href{https://richmondalake.medium.com/yolo-nas-uncovered-essential-insights-and-implementation-techniques-for-machine-learning-engineers-87ee266b37f6}{\textcolor{blue}{Source}}}
    \label{fig:enter-label}
\end{figure}
\href{https://deci.ai/blog/yolo-nas-object-detection-foundation-model/}{\textcolor{blue}{Source-1}}
\href{https://deci.ai/model-zoo/yolo-nas/}{\textcolor{blue}{Source-2}}  



\subsubsection{Architectural Features of YOLO-NAS} \vspace{0mm}It is considered one of the most efficient object detection algorithms due to its unique architecture, which is designed to minimize the computational cost of the algorithm while maintaining high accuracy. \cite{yolo-nas-vs-yolov8}[\href{https://deci.ai/blog/yolov8-vs-yolo-nas-showdown-exploring-advanced-object-detection/#:~:text=Quantization%20Aware%20Blocks%20and%20Selective%20Quantization&text=This%20architecture%20utilizes%20Quantization%2DSpecific,parameterization%20with%208%2Dbit%20quantization.}{\textcolor{blue}{Source: 1}}]   
\begin{enumerate}
    \item \textbf{Quantization Aware Blocks and Selective Quantization: } 
        \begin{itemize}
            \item \textbf{Hybrid Quantization Method: } YOLO-NAS employs a hybrid quantization strategy, integrating Quantization-Specific Parameters (QSP) and Quantization-Centric Initialization (QCI) blocks. These leverage re-parameterization and 8-bit quantization, inspired by Chu et al.'s methodology.
            \item  \textbf{Selective Quantization:} The model strategically quantizes specific parts rather than uniformly affecting all layers. This selective quantization balances accuracy and latency, addressing information loss, a common issue in standard quantization techniques.
                \item \textbf{Layer Selection Algorithm:} YOLO-NAS utilizes a sophisticated layer selection algorithm to decide which layers to quantize. It evaluates the impact of each layer on accuracy and latency, carefully considering the implications of toggling between 8-bit and 16-bit quantization.
                \item \textbf{Performance in Limited Resources:} Uniquely designed for environments with limited resources, YOLO-NAS ensures superior performance. Its optimized approach to quantization preserves accuracy while being efficient, striking a crucial balance for advanced object detection tasks.
        \end{itemize}
    \item \textbf{Detection Head: } A standout feature of YOLO-NAS is its detection head design, which predicts a distribution probability for size regression. This approach is particularly useful in scenarios where the object sizes vary significantly. By predicting a range of possible sizes rather than a single fixed size, YOLO-NAS enhances its accuracy in detecting objects of varying scales. This probabilistic approach to size regression also makes YOLO-NAS suitable for knowledge distillation. It allows for a more nuanced transfer of knowledge from a complex, high-capacity teacher model to a simpler, more efficient student model, further enhancing the practicality of YOLO-NAS in diverse applications.
    \item \textbf{Neck: } The YOLO-NAS network has a highly advanced pyramid-attention neck that combines both top-down and bottom-up information flows. 
    \begin{figure}[H]
        \centering
        \includegraphics[width=\textwidth]{yolo-nas_neck.PNG}
        \caption{Neck Design}
        \label{fig:Neck}
    \end{figure}
    This design element assists the network in effectively capturing and utilizing multi-scale information. In this pyramid-attention structure, the top-down pathway captures high-level semantic information, while the bottom-up pathway focuses on smaller details. The attention mechanism in this pyramid structure ensures that the network focuses on the most relevant features at different scales, which improves its ability to detect objects with varying sizes and complexities. 
    \item \textbf{Backbone: } 
The backbone of YOLO-NAS, a pivotal component in its architecture, is a product of an advanced Network Architecture Search (NAS) process, specifically utilizing Deci’s proprietary NAS technology, AutoNAC. This innovative approach enables the tailoring of the network structure to meet the specific demands of object detection tasks with unparalleled precision.
\begin{itemize}
    \item \textbf{quantization-aware RepVGG: } During the NAS process, they have incorporated quantization-aware RepVGG blocks into the model architecture, ensuring that our model architecture would be compatible with Post-Training Quantization (PTQ).\\
    \item \textbf{Spatial Pyramid Pooling (SPP): } YOLO-NAS backbone has Spatial Pyramid Pooling (SPP)  at the end to capture global context, it is a pooling layer that removes the fixed-size constraint of the network, i.e. a CNN does not require a fixed-size input image. Placing SPP at the end of the YOLO-NAS backbone allows the model to capture global context information, which is crucial for understanding the overall context of the scene. This can be beneficial for detecting objects that may span a larger region in the image.
\end{itemize}

\href{https://deci.ai/blog/yolo-nas-object-detection-foundation-model/}{Source}

\begin{figure}[H]
    \centering
    \includegraphics[width=\textwidth]{Yolo-BackBone.PNG}
    \caption{Backbone Structure}
    \label{fig:BackBone}
\end{figure}

AutoNAC, as a part of the NAS process, plays a critical role in determining the optimal sizes and structures of various stages within the YOLO-NAS backbone. This includes meticulously configuring the block type, the number of blocks, and the number of channels in each stage. By doing so, AutoNAC ensures that each aspect of the backbone is precisely optimized for its function.

The NAS-generated backbone, therefore, is not just a result of automated design but also of intelligent decision-making that considers a multitude of factors. This includes computational efficiency, accuracy, and speed, ensuring that the backbone contributes effectively to the overall robustness and adaptability of YOLO-NAS.

These advancements culminate in a superior architecture with unprecedented object detection capabilities and outstanding performance over its predecessors.
\end{enumerate} 

\subsubsection{Automated Neural Architecture Construction (AutoNAC) engine}
Deci-AI utilizes an optimization engine called AutoNAC, which was developed by Deci. This engine applies Neural Architecture Search (NAS) to enhance the architecture of a deep learning model. The main goal is to improve the performance of the model when it is executed on specific hardware while maintaining or even improving its accuracy. The AutoNAC engine is hardware aware, data-aware and considers all the components in the inference stack, including compilers and quantization. \\
AutoNAC is essentially a NAS engine takes three components as inputs. \cite{yolo-nas-webinar}
\begin{itemize}
    \item \textbf{Task: } Firstly, we need to decide which model to build, such as an object detection model.\\
    \item \textbf{Data characteristics: } Deci Group requires the NAS Engine to be data-aware or capable of understanding the characteristics of the data so that it can create a model that is well-suited for the task at hand. For instance, if the objective is to detect small objects, the model would likely be different from that required for detecting large objects, as the receptive fields would differ. \\
    \item \textbf{Inference environment and Hardware: } When it comes to computer vision on edge, the priority is to achieve real-time performance or improve some level of latency or throughput while being hardware-aware, compilation-aware, and quantization-aware. To achieve this, we need to consider factors such as model size, latency, and throughput. Decis Group's goal is to optimize these factors as part of the Neural Architecture Search (NAS) process. Manually finding the optimal architecture for object detection models can be a laborious and inefficient process. To address this issue, Deci utilized AutoNAC, a tool that uses neural design space incorporating state-of-the-art (SOTA) architectural design principles and Deci's novel neural elements, to discover novel object detection models. These models were optimized to minimize inference latency computed over NVIDIA's T4 cloud GPU, which is a widely used computing device. We achieve this by feeding all three inputs into the AutoNAC engine, which generates a new architecture.
\end{itemize}
 
\begin{figure}[H]
    \centering
    \includegraphics[width=\textwidth]{autoNAC_Engine.PNG}
    \caption{Deci's AutoNAC Engine; \\
    Hardware-Aware Neural Architecture Search for DL Inference Efficiency}
    \label{fig:AutoNAC Engine}
\end{figure}

\subsubsection{Under the hood View of Auto-NAC}
NAS algorithms can systematically explore the vast search space of potential architectures, effectively identifying novel and optimized configurations that might be overlooked by human intuition. By automating the process, these algorithms can efficiently evaluate and compare a vast number of candidate architectures with an unfathomable search space of $10^{14}$ possible architectures, ultimately converging on a solution that optimally balances accuracy, speed, and complexity, and the NAS engine comes into place when we have data already and want to build a model.
it takes all three above-mentioned inputs, and the search space is automatically created under the hood taking into account all the possibilities of Neural Architecture as an input. For Example, opti mal sizes, block types, number of blocks, and channel counts in every stage. \cite{yolo-nas-webinar}  \href{https://deci.ai/blog/sota-dnns-overview/}{\textcolor{blue}{Source}}  
\begin{figure}[H]
    \centering
    \includegraphics[width=\textwidth]{AUTONACENGINE_Underthe hood.PNG}
    \caption{AutoNAC Engine Search Space}
    \label{fig:autonac_under_the_hood}
\end{figure}
Finally, AutoNAC Engine explores and maps the efficiency frontier, searching for an architecture that best balances latency vs. throughput. Deci Group samples three points of this frontier to create the YOLO-NASS, YOLO-NASM, and YOLO-NASL architectures.
\subsection{How to find the best architecture out of the search space}
while the search space has trillions of candidate architectures, the task of identifying the optimal architecture becomes increasingly challenging. Within the realm of Deci, the selection of the most suitable candidate from the numerous models within the extensive search space is guided by two distinct evaluation criteria. \cite{yolo-nas-webinar}
\begin{enumerate}
    \item \textbf{Bench Marking: } Understanding how the model will behave on the expected hardware or using the techniques of Latency Vs. throughput for the candidate architecture. \\
    \item \textbf{Accuracy Potential Predictions: } The fundamental selection methodology of best candidate out of the search space involves the provision of an AI model, designated to assess the accuracy or potential accuracy of neural network models. Leveraging principles from neural architecture search, YOLO-NAS systematically navigates the expansive design space of possible neural network configurations, employing optimization algorithms to identify architectures anticipated to yield high accuracy on the specified object detection task.
\end{enumerate}
 \begin{figure}[H]
     \centering
     \includegraphics[width=\textwidth]{Yolo_nas.png}
     \caption{\textbf{YOLO-NAS Architecture. The architecture is found automatically via a Neural Architecture Search(NAS) system called AutoNAC to balance latency vs. throughput. They generated three architecture called YOLO-NASS(small), YOLO-NASM(medium), and YOLO-NASL(large), varying the depth and the position of the QSP and QCI blocks \cite{YOLO-NAS}}}
     \label{fig:YOLO-NAS Architecture.}
\end{figure}


\subsubsection{Fine-tuning of the selected model On custom data set}
Fine-tuning is the process of making small adjustments to achieve the desired output or performance. In deep learning, it involves the use of weights of a trained neural network to program another deep learning algorithm from the same domain. The weights connect each neuron in one layer to every neuron in the next layer in the neural network. Fine tuning is used to speed up the training and/or overcome a small dataset as it already contains vital information from a pre-existing deep learning algorithm. The YOLO-NAS architecture and pre-trained weights define a new frontier in low-latency inference and an excellent starting point for fine-tuning downstream tasks. Strong pre-trained weights often lead to higher model accuracy on new datasets when fine-tuning. YOLO-NAS was trained on the RoboFlow100 dataset (RF100), a collection of 100 datasets from diverse domains, to demonstrate its ability to handle complex object detection tasks. The RF100 dataset is a benchmark for existing YOLO models, enabling us to compare YOLO-NAS’s performance against them and showcase its advantages.\cite{YOLO-NAS}
\begin{figure}[H]
    \centering
    \includegraphics[width=0.8\textwidth]{YOLO-NAS_RF100_benchmark.PNG}
    \caption{Examples of annotated images in the RF100 benchmark}
    \label{fig:YOLO-NAS_RF100_benchmark}
\end{figure}
The following hyperparameters ensure a robust and consistent training process, allowing for a fair comparison of the model’s performance across different datasets.\\
Deci followed the RF100 repository’s training protocol to ensure a fair comparison.\\
The training protocol for the model includes the following settings, applied consistently across all datasets for 100 epochs on a single T4 GPU with 16GB of VRAM during training:\cite{YOLO-NAS}, \cite{yolo-nas-vs-yolov8}
\begin{itemize}
    \item Learning rate: 5e-4 for the Small version and 4e-4 for the Medium version of the model
    \item Weight decay: 1e-4 (excluding bias and BatchNorm layers)
    \item Exponential moving average (EMA) with a decay factor of 0.99
    \item Batch size: 16
    \item Image resolution: 640×640
\end{itemize}


\begin{figure}[H]
  \begin{minipage}{0.48\textwidth}
    \centering
    \includegraphics[width=\linewidth]{YOLO-NASSM_vs_other_models.PNG}
    \caption{Average mAP on Roboflow-100 \\for YOLO-NAS vs other models.}
    \label{fig:YOLO-NASSM_vs_other_models}
  \end{minipage}%
  \begin{minipage}{0.5\textwidth}
    \centering
    \includegraphics[width=\linewidth]{YOLO-NAS_vs_other_models.PNG}
    \caption{Per category mAP score for YOLO-NAS vs other models.
    Note: For Yolo vV5/v7/v8}
    \label{fig:YOLO-NAS_vs_other_models}
  \end{minipage}
  \caption{Average mAP and Per category mAP score for YOLO-NAS vs other models.}
\end{figure}
Figure \ref{fig:YOLO-NASSM_vs_other_models} are the results obtained by focusing on the “Small” and “Medium” YOLO-NAS variants, and figure: \ref{fig:YOLO-NAS_vs_other_models} is a per-category breakdown of YOLO-NAS’s performance on the RF-100 dataset, compared to the performance of v5/v7/v8 models. \cite{YOLO-NAS}, \cite{yolo-nas-v8-sota}
\href{https://deci.ai/blog/yolo-nas-object-detection-foundation-model/}{\textcolor{blue}{Source}}
\subsubsection{Object Detection Evaluation Metrics}
Object detection metrics are measures used to evaluate the performance of object detection algorithms
Performance metrics are key tools to evaluate the accuracy and efficiency of object detection models. They shed light on how effectively a model can identify and localize objects within images. Additionally, they help in understanding the model's handling of false positives and false negatives. These insights are crucial for evaluating and enhancing the model's performance.
There are many evaluation metrics and the following are broadly applicable accross different object detection, also specifically in YOLO-NAS model. \cite{metrics}
\href{https://docs.ultralytics.com/guides/yolo-performance-metrics/#object-detection-metrics}{\textcolor{blue}{Source}}
\begin{itemize}
    \item \textbf{Intersection over Union (IoU):} IoU is a measure that quantifies the overlap between a predicted bounding box and a ground truth bounding box. It plays a fundamental role in evaluating the accuracy of object localization.
    \item \textbf{Average Precision (AP):} AP computes the area under the precision-recall curve, providing a single value that encapsulates the model's precision and recall performance.
    \item \textbf{Mean Average Precision (mAP):} mAP extends the concept of AP by calculating the average AP values across multiple object classes. This is useful in multi-class object detection scenarios to comprehensively evaluate the model's performance.
    \item \textbf{Precision and Recall:} Precision quantifies the proportion of true positives among all positive predictions, assessing the model's capability to avoid false positives. On the other hand, Recall calculates the proportion of true positives among all actual positives, measuring the model's ability to detect all instances of a class.
    \item F1 Score: The F1 Score is the harmonic mean of precision and recall, providing a balanced assessment of a model's performance while considering both false positives and false negatives.
\end{itemize}

\subsubsection{Benefits of YOLO-NAS over other models}
\textbf{Optimized Efficiency: }\\
YOLO-NAS excels in achieving an optimal balance between accuracy and speed, surpassing other human-designed models in terms of efficiency. This optimization plays a pivotal role in enhancing inference speeds and improving resource utilization, which are essential for the demands of real-time object detection applications.\\
\textbf{Adaptability to Diverse Tasks and Hardware: }\\
YOLO-NAS’s architecture makes it adaptable to a variety of object detection tasks, ranging from common objects to more intricate scenarios and hardware. This versatility positions YOLO-NAS as a valuable tool across a spectrum of applications.\\
\textbf{Training on Prominent Datasets: }\\
YOLO-NAS enhances its capabilities through pre-training on the well established Object365, followed by training on COCO. This strategic pre-training equips the model with a comprehensive understanding of common object features, significantly boosting its accuracy when presented with new and previously unseen images.\\
\textbf{Detection and Localization Accuracy: }\\
YOLO-NAS specialized architecture adeptly addresses challenges associated with small or intricate objects, providing not only heightened detection capabilities but also improved accuracy in pinpointing precise locations. These advancements position YOLO-NAS as a superior choice for a diverse range of applications, especially those where identifying small or elusive objects is pivotal. While YOLOv8 is an impressive object detection model, it encounters limitations when tasked with the intricate challenge of small object detection and localization accuracy. Although proficient in various scenarios, YOLOv8 falls short compared to the specialized capabilities of YOLO-NAS in handling challenges posed by diminutive objects. \cite{yolo-nas-vs-yolov8}
\begin{figure}[H]
    \centering
    \includegraphics[width=0.7\textwidth]{YOLO-V8_vs_YOLO-NAS.PNG}
    \caption{Performance evaluation on YOLO-V8 vs. YOLO-NAS \href{https://www.youtube.com/watch?v=uPgE8G4CGF4}{\textcolor{blue}{Source}}}
    \label{Yolo-V8_vs_YOLO-NAS}
\end{figure}


\subsubsection{Data set}
The "p4p Cars Computer Vision Project" stands as a meticulously curated dataset within the Roboflow Universe, purposefully designed for object detection, with a specific emphasis on vehicular recognition. Encompassing a diverse array of vehicle types, including buses, cars, motorbikes, trucks, and vans, this dataset plays a crucial role in advancing computer vision applications.

In terms of key attributes, the project adopts an object detection framework with distinct classes such as bus, car, motorbike, truck, and van. The dataset comprises 10,000 images, with data allocation distributed as 70\texttt{\%} for training, 20\texttt{\%}for validation, and 10\texttt{\%} for testing.

Data preprocessing involves the application of Auto-Orient, resizing images to 640x640 with a stretch mechanism, and modifying classes with 0 remapped and 0 dropped instances. Notably, no augmentations were applied during the preprocessing phase.

This dataset holds substantial utilitarian significance across various domains. It facilitates real-time traffic monitoring and analysis, contributing to the refinement of overarching traffic management strategies. Additionally, it enables security entities to identify and catalog vehicles entering or exiting specific zones. The dataset serves as a valuable reservoir of training data for autonomous driving technologies, aiding self-driving vehicles in discerning between distinct vehicle classes on road networks. Enterprises relying on vehicular fleets, encompassing rental, logistics, or transportation sectors, can leverage the dataset for automatic classification, optimizing inventory management protocols. Moreover, insurance and law enforcement agencies find utility in expediting the analysis of accident scenes, where automatic identification of involved vehicle classes streamlines claims adjudication and investigative procedures.

The "p4p Cars Computer Vision Project" dataset epitomizes substantive potential across domains, ranging from traffic management and security to autonomous vehicles, fleet management, and traffic accident analysis.\cite{p4p-cars_dataset}

 % zazwyczaj wystarczy podmienić plik src/wut-thesis.cls
\section{Object detection Implementations and fine-tuning on custom dataset}

\begin{enumerate}
    \item \textbf{Environment Setup: } Google colab is free resaerch environment provided by $Google^{Tm}$. Using Google Colab for training deep learning models, including YOLO-NAS, provides you with access to free GPU resources. Before we start training, we need to prepare our Python environment and start by installing the required packages.
    The YOLO-NAS model is distributed using a \boxed{\textcolor{black}{super-gradients}}  package. In addition, we will install \boxed{roboflow} and \boxed{supervision}, which will allow us to download the dataset from Roboflow Universe and visualize the results of our training respectively. We use \boxed{\texttt{!pip 
    install}} to install the required package. 1. \href{https://deci.ai/blog/how-to-train-yolo-nas-with-supergradients-a-step-by-step-guide/}{\textcolor{blue}{Source 1}},
2.\href{https://blog.roboflow.com/yolo-nas-how-to-train-on-custom-dataset/#what-is-yolo-nas}{\textcolor{blue}{Source}} \\

3.\href{https://docs.deci.ai/super-gradients/latest/YOLONAS.html#quickstart}{\textcolor{blue}{quick start}}\\
    
    \begin{lstlisting}[language=Python, caption=Installing the required package]
    !pip install super-gradients
    !pip install roboflow
    !pip install supervision
        
    \end{lstlisting}
    \item \textbf{Load YOLO-NAS Model}
    To support the open-source community, Deci group has released the capability to fine-tune YOLO-NAS models on the RF100 dataset within the SuperGradients library. To perform inference using the pre-trained model, we first need to choose the size of the model. YOLO-NAS offers three different model sizes: \verb|yolo_nas_s|, \verb|yolo_nas_m|, and \verb|yolo_nas_l|.
    \begin{lstlisting}[language=Python, caption=importing models from the supergradients][H]
    from super_gradients.training import models

    model = models.get(yolo_nas_l, pretrained_weights="coco").to(DEVICE)
    \end{lstlisting}
    \item \textbf{Dataset Download and Directory Structure: }
   To perform model fine-tuning, the acquisition of pertinent data is imperative. The requisite dataset structure for this endeavor adheres to the YOLO format. Presently, I am utilizing a dataset sourced from the Roboflow universe, encompassing a vast repository of over 100,000 open-source computer vision datasets. Significantly, the dataset obtained from Roboflow is inherently organized to align with the prescribed folder structure mandated by YOLO standards.

    \begin{lstlisting}[caption=Dataset folder structure YOLO format][style=custom,language=sh]
    /directory_to_your_dataset
    |-- train
    |   |-- images
    |   |   |-- image1.jpg
    |   |   |-- image2.jpg
    |   |   |-- ...
    |   |-- labels
    |       |-- image1.txt
    |       |-- image2.txt
    |       |-- ...
    |-- valid
    |   |-- images
    |   |-- labels
    |-- test
    |   |-- images
    |   |-- labels
    \end{lstlisting}
    Each image file should have a corresponding label file in the YOLO format, which contains one line per object in the image. Each line should have the following format:
    \texttt{<class\_id> <x\_center> <y\_center> <width> <height>}
    
    where \verb|<class_id>| is an integer representing the class of the object, and 
    \texttt{<x\_center>, <y\_center>, <width>, and <height>}
    are the normalized coordinates of the bounding box.
   In the process of importing data from the Roboflow universe, the inclusion of a data.yaml file is an automated or default feature. This file serves as a dictionary, encapsulating the names of classes associated with the imported data.  
    The order of the class names should match the \texttt{<class\_id>} in the label files.
    \item \textbf{Dataset Parameters for Training YOLO NAS}
    After preparing your data, you need to create a Python dictionary that outlines the key elements of your dataset’s structure, such as: \href{https://learnopencv.com/train-yolo-nas-on-custom-dataset/}{\textcolor{blue}{Source}}
    \begin{lstlisting}[language=Python]
        dataset_params = {
        'data_dir': LOCATION,
        'train_images_dir': 'train/images',
        'train_labels_dir': 'train/labels',
        'val_images_dir': 'valid/images',
        'val_labels_dir': 'valid/labels',
        'test_images_dir': 'test/images',
        'test_labels_dir': 'test/labels',
        'classes': CLASSES
    }
    \end{lstlisting}
    Using the \texttt{dataset\_params} we can easily create the datasets in the required format later on,   Where LOCATION is the path to the main directory where your dataset resides, and CLASSES is the list of class names you created earlier. Also, we need to define the number of training epochs, the batch size, and the number of workers for data processing.
    \begin{lstlisting}[language=Python]
        EPOCHS = 50,
        BATCH_SIZE = 16,
        WORKERS = 8 
    \end{lstlisting}

  
    
    \item \textbf{Setting-up dataloaders: }  SuperGradients provides ready-to-use dataloaders for the datasets it supports. Import required modules for SuperGradients dataloaders.
    \begin{lstlisting}[language=Python, caption=Setting-Up Dataloaders]
    from super_gradients.training import dataloaders
    from super_gradients.training.dataloaders.dataloaders import
    (coco_detection_yolo_format_train, coco_detection_yolo_format_val) 
    \end{lstlisting}
     Create data loaders for training, validation, and testing sets with specified batch size and number of workers. Below we use \texttt{batch\_size}=16 and \texttt{num\_workers}=2. Note that for training and testing data we use \verb|coco_detection_yolo_format_val| to instantiate the dataloader.  
     \begin{lstlisting}[language=Python, caption=Setting-Up Dataloaders]
        from IPython.display import clear_output 
        train_data = coco_detection_yolo_format_train(
            dataset_params={
                'data_dir': dataset_params['data_dir'],
                'images_dir': dataset_params['train_images_dir'],
                'labels_dir': dataset_params['train_labels_dir'],
                'classes': dataset_params['classes']
            },
            dataloader_params={
                'batch_size': BATCH_SIZE,
                'num_workers': 2
            }
        )
        
        val_data = coco_detection_yolo_format_val(
            dataset_params={
                'data_dir': dataset_params['data_dir'],
                'images_dir': dataset_params['val_images_dir'],
                'labels_dir': dataset_params['val_labels_dir'],
                'classes': dataset_params['classes']
            },
            dataloader_params={
                'batch_size': BATCH_SIZE,
                'num_workers': 2
            }
        )
        
        test_data = coco_detection_yolo_format_val(
            dataset_params={
                'data_dir': dataset_params['data_dir'],
                'images_dir': dataset_params['test_images_dir'],
                'labels_dir': dataset_params['test_labels_dir'],
                'classes': dataset_params['classes']
            },
            dataloader_params={
                'batch_size': BATCH_SIZE,
                'num_workers': 2
            }
        )
        clear_output()
     \end{lstlisting}
     \item \textbf{Model Instantiation for Fine-Tuning}
     Choose the YOLO-NAS variant (e.g., \texttt{yolo\_nas\_l}) and instantiate the model, specifying the number of classes based on your dataset.
     \begin{lstlisting}
        from super_gradients.training import models
        model = models.get('yolo_nas_l', num_classes=len(dataset_params
        ['classes']), pretrained_weights="coco")
     \end{lstlisting}
    \item \textbf{Training Parameters and Metrics: } When setting up your training using SuperGradients, it's important to define your training parameters carefully. These parameters not only affect your training process, but also have a significant impact on the performance of your model. To help you get started, here's a brief overview of the most crucial parameters and some advanced features you should consider. \\
    
    \textbf{Essential Training Parameters}\\
    \begin{itemize}
        \item \textbf{\texttt{max\_epochs}}: Sets the maximum number of training cycles. \\
        \item \textbf{loss: } Choose a suitable loss function for your model and dataset. \\
        \item \textbf{optimizer: } Opt for an optimizer (e.g., Adam, AdamW, SGD) and customize it if necessary through \texttt{optimizer\_params}.\\
        \item \textbf{\texttt{train\_metrics\_list}/\texttt{valid\_metrics\_list}: } Specify metrics for evaluating training and validation performance. These are implemented as \href{https://lightning.ai/docs/torchmetrics/stable/}{Torchmetric} objects. \\
        \item \textbf{\texttt{metric\_to\_watch}: } Select a primary metric to determine checkpoint saving. \\
    \end{itemize} 
    
    \textbf{Advanced Features and Integrations}\\
    \begin{itemize}
        \item \textbf{Monitoring Tools: } Utilize integrations with Tensorboard, Weights and Biases, or ClearML for tracking training progress. Custom integrations can be achieved using Base SGLogger.
        \item \textbf{Training Enhancements: } SuperGradients offers features like Exponential Moving Average, Zero Weight Decay on Bias and Batch Normalization, Weight Averaging, Batch Accumulation, and Precise BatchNorm for optimized training.
        \item \textbf{Custom Metrics: } SuperGradients supports the creation of custom metrics for specialized needs.
    \end{itemize}
    
    \textbf{Loss and Detection Metrics Setup: }\\
    For functions like PPYoloELoss and \texttt{DetectionMetrics\_050}, ensure you specify the number of classes in your dataset correctly to align loss calculations and metrics evaluation with your dataset’s specifics.
    
    \begin{lstlisting}[language=Python]
        from super_gradients.training.losses import PPYoloELoss
        from super_gradients.training.metrics import DetectionMetrics_050
        from super_gradients.training.models.detection_models.pp_yolo_e 
         import PPYoloEPostPredictionCallback
        
        
        train_params = {
           # ENABLING SILENT MODE
           'silent_mode': True,
           "average_best_models":True,
           "warmup_mode": "linear_epoch_step",
           "warmup_initial_lr": 1e-6,
           "lr_warmup_epochs": 3,
           "initial_lr": 5e-4,
           "lr_mode": "cosine",
           "cosine_final_lr_ratio": 0.1,
           "optimizer": "Adam",
           "optimizer_params": {"weight_decay": 0.0001},
           "zero_weight_decay_on_bias_and_bn": True,
           "ema": True,
           "ema_params": {"decay": 0.9, "decay_type": "threshold"},
           # ONLY TRAINING FOR 10 EPOCHS FOR THIS EXAMPLE NOTEBOOK
           "max_epochs": 10,
           "mixed_precision": True,
           "loss": PPYoloELoss(
               use_static_assigner=False,
               # NOTE: num_classes needs to be defined here
               num_classes=len(dataset_params['classes']),
               reg_max=16
           ),
           "valid_metrics_list": [
               DetectionMetrics_050(
                   score_thres=0.1,
                   top_k_predictions=300,
                   # NOTE: num_classes needs to be defined here
                   num_cls=len(dataset_params['classes']),
                   normalize_targets=True,
                   post_prediction_callback=PPYoloEPostPredictionCallback(
                       score_threshold=0.01,
                       nms_top_k=1000,
                       max_predictions=300,
                       nms_threshold=0.7
                   )
               )
           ],
           "metric_to_watch": 'mAP@0.50'
        }
    \end{lstlisting}
    \item \textbf{Instantiate the training: }
    \begin{lstlisting}[language=Python, caption=Commencing the Training, escapeinside=``]
        trainer.train(model=model, training_params=train_params,
        train_loader=train_data, valid_loader=val_data)
    \end{lstlisting}
    \item \textbf{ Fetching the Best(Optimal) Model after Training: } Following intensive training, wherein the model underwent rigorous learning iterations, we are now focused on retrieving the best-trained model. This phase involves selecting the most refined and proficient version of the model, one that has successfully absorbed and generalized patterns from the training data. The goal is to identify the model that exhibits superior performance, ensuring its readiness for deployment in real-world scenarios. This meticulous retrieval process is crucial for obtaining a highly capable and reliable model that aligns with the desired objectives and expectations. In SuperGradients, this involves selecting the most effective set of weights from your training runs.\\
    \textbf{Using Checkpoint Averaging}
    \begin{addmargin}[8mm]{0mm}
    \begin{lstlisting}[language=Python, caption=Load trained model][H]
    from super_gradients.training import models

    # Path to the averaged checkpoint
    averaged_checkpoint_path = 'checkpoints/my_first_yolonas_run
                    /average_model.pth'
    
    # Retrieve the model with averaged weights
    best_model = models.get('yolo_nas_l', 
                            num_classes=len(dataset_params['classes']),
                            checkpoint_path=averaged_checkpoint_path)
        
    \end{lstlisting}
    \end{addmargin}
    \item \textbf{Evaluate trained model}
    \begin{addmargin}[8mm]{0mm}
    \begin{lstlisting}[language=Python, caption=Model Evaluation on Test Data][H]
        trainer.test(
            model=best_model,
            test_loader=test_data,
            test_metrics_list=DetectionMetrics_050(
                score_thres=0.1,
                top_k_predictions=300,
                num_cls=len(dataset_params['classes']),
                normalize_targets=True,
                post_prediction_callback=PPYoloEPostPredictionCallback(
                    score_threshold=0.01,
                    nms_top_k=1000,
                    max_predictions=300,
                    nms_threshold=0.7
                )
            )
        )
    \end{lstlisting}
    \end{addmargin}
    \item \textbf{Model Evaluation on Test Data: } Evaluating a model on test data is a pivotal phase within the machine learning workflow, serving to gauge the effectiveness of a trained model on data it has not encountered during training. The primary objective is to ascertain the model's generalization capability to novel, unseen instances. This section outlines prevalent metrics and procedures employed in the assessment of a model using test data.
\begin{addmargin}[8mm]{0mm} 
\begin{lstlisting}[
    language=Python,
    caption=Inference with trained model,
    numbers=left,
    firstnumber=1,
    aboveskip=10pt,
    xleftmargin=10mm,  % Adjust the left margin as needed
    linewidth=\linewidth,  % Ensures that the code fits within the page width
]
img_url = 'https://www.mynumi.net/media/catalog/product/cache/2/
image/9df78eab33525d08d6e5fb8d27136e95/s/e/serietta_usa_2_1/
www.mynumi.net-USASE5AD160-31.jpg'
best_model.predict(img_url).show()     
\end{lstlisting}
\end{addmargin}


    

\end{enumerate}
\section{Result and discussion}
The YOLO-NAS has YOLO-NAS-L, YOLO-NAS-M, and YOLO-NAS-S variants, and in the result and discussion I am tailored to discuss individual performance characteristics of each model, and every model has three paths to choose from; the best model, average model, and latest model, and particularly I performed inference using the best weight of each model.

\subsection{Training and Implementation Details}
A batch size of 16 was used for all models, and they were trained for 60 epochs on NVIDIA Tesla T4 GPU with High-RAM. Before commencing training, various hyperparameters were adjusted. The "average best models" option was enabled, and warm-up models were set up with a linear epoch step. The initial learning rate was set to 1e-6 during warm-up, and the learning rate decay factor was set to 3. The initial learning rate was set to 5e-4, and the learning rate decay mode was cosine. The cosine-final learning rate ratio was set to 0.1, and the optimizer was Adam. The weight decay in optimizer parameters was set to 0.0001, and no weight decay was applied to bias and batch-normalization.
Furthermore, exponential moving averaging was used with a decay factor of 0.9, and the decay type was set to a threshold. The "mixed precision" option was enabled during training. All models were implemented and trained on Google Colab.
\subsubsection{YOLO-NAS-S}
The performance evaluation of inference on both models(i.e., before training and after training) is conducted utilizing a metric known as Intersection over Union (IoU) with a value of 0.5 and above and a confidence level of 0.5 and above. The images used to evaluate the model's performance are foreign images.
\begin{figure}[H]
  \centering
  \begin{subfigure}{\textwidth}
    \centering
    \includegraphics[width=\textwidth]{tex/img/YNS2_BT_3.png}
    \caption{YOLO-NAS-S: Inference on the Model before training}
    \label{fig:sub_s1}
  \end{subfigure}
  \hfill
  \begin{subfigure}{\textwidth}
    \centering
    \includegraphics[width=\textwidth]{tex/img/YNS2_AT_3.png}
    \caption{YOLO-NAS-S: Inference on the Model after training}
    \label{fig:sub_s2}
  \end{subfigure}
  \caption{YOLO-NAS: Inference on the model before and  after training .}
  \label{fig:NAS-S} 
\end{figure}

Both models exhibit almost the same level of performance, but as can be seen from both images, the trained model outperforms with a slight difference where it shows a slightly better trade-off between precision and recall, also the dataset used for training is annotated mostly for not distant or short range, as a consequence the model is also exhibiting that it is struggling to detect the distant classes, sometimes the fine-tuned model is good at detecting distant objects.


 \begin{figure}[H]
    \centering
    \includegraphics[width=0.5\linewidth]{tex/img/YNS2_AP_3.png}
    \caption{Visualization of the performance of YOLO-NAS-S with bounding boxes and corresponding obstacle
classes predicted by the model.}
    \label{fig:S-annot-pred}
\end{figure}

\begin{figure}[H]
The Confusion Matrix illustrates the effectiveness of the overall YOLO-NAS-S in detecting obstacles across four distinct classes. Notably, the matrix reveals a notable variation in the true positive values. Specifically, the highest true positive rate, standing at 1 which is not good, is achieved in detecting the 'motorbike' class. Conversely, the 'truck' class exhibits the lowest true positive rate, measured at 0.55.
    \centering
    \includegraphics[width=\linewidth]{tex/img/YNS2_CM.png}
    \caption{YOLO-NAS-S Overall results }
    \label{fig:ConfusionMatrixY-N-S}
\end{figure}
%%%%%%%%%%%%%%%%%%%%%%%%
%YOLONAS-M
%%%%%%%%%%%%%%%%%%%%
\newpage
\subsubsection{YOLO-NAS-M}
The performance evaluation of inference for models is the same for all. As shown in \ref{fig:sub_M1}, The pre-trained model produces false positives, and after extensive training, the trained model exhibits very good results in detecting the short-range objects as YOLO-NAS-S model because of the dataset properties.
\begin{figure}[hpbt]
  \centering
  \begin{subfigure}{\textwidth}
    \centering
    \includegraphics[width=0.9\textwidth]{tex/img/YNM2_BT_6.png}
    \caption{YOLO-NAS-M: Inference on the Model before training}
    \label{fig:sub_M1}
  \end{subfigure}
  \hfill
  \begin{subfigure}{\textwidth}
    \centering
    \includegraphics[width=0.9\textwidth]{tex/img/YNM2_AT_6.png}
    \caption{YOLO-NAS-M: Inference on the Model after training}
    \label{fig:sub_M2}
  \end{subfigure}
  \caption{YOLO-NAS-M: Inference on the model before and  after training .}
  \label{fig:NAS-M} 
\end{figure}



\begin{figure}[H]
    \centering
    \includegraphics[width=0.6\linewidth]{tex/img/YNM2_AP_4.png}
    \caption{Visualization of the performance of YOLO-NAS-M with bounding boxes and corresponding obstacle
classes predicted by the model.}
    \label{fig:M-annot-pred}
\end{figure}

\begin{figure}[H]
The Confusion Matrix illustrates the effectiveness of the overall YOLO-NAS-M in detecting obstacles across four distinct classes. Notably, the matrix reveals a notable variation in the true positive values. Specifically, the highest true positive rate, standing at 0.93 which is good, is achieved in detecting the 'motorbike' class. Conversely, the 'truck' class exhibits the lowest true positive rate, measured at 0.56 which is better than the small model, from this we can tell that it is slightly better at balancing the trade-off between recall and precision, this is due to the size of the dataset in the backbone of the pre-trained model. The table shows that the model is losing to detect a car class or generates a significant number of false negatives for this category. This is likely attributed to the model's struggle in detecting distant cars.
    \centering
    \includegraphics[width=\linewidth]{tex/img/YNM2_CM.png}
    \caption{YOLO-NAS-M Overall results }
    \label{fig:ConfusionMatrixY-N-M}
\end{figure}

%%%%%%%%%%%%%%%%%%%%%%%%%%%%%%
%YOLO-NAS-L
%%%%%%%%%%%%%%%%%%%%%%%%%%
\subsubsection{YOLO-NAS-L}
The performance evaluation of inference for models is the same for all.
As the model size of the pre-trained model increases, the model exhibits an increase in the trade-off between precision and recall and I will discuss extensively in Table: \ref{tab:my_label} 
\begin{figure}[H]
  \centering
  \begin{subfigure}{\textwidth}
    \centering
    \includegraphics[width=\textwidth]{tex/img/YNL2__BT_6.png}
    \caption{YOLO-NAS-L: Inference on the Model before training}
    \label{fig:sub_M1}
  \end{subfigure}
  \hfill
  \begin{subfigure}{\textwidth}
    \centering
    \includegraphics[width=\textwidth]{tex/img/YNL2__AT_6.png}
    \caption{YOLO-NAS-L: Inference on the Model after training}
    \label{fig:sub_M2}
  \end{subfigure}
  \caption{YOLO-NAS-L: Inference on the model before and  after training .}
  \label{fig:NAS-L} 
\end{figure}


\begin{figure}[H]
    \centering
    \includegraphics[width=0.6\linewidth]{tex/img/YNL2_Ann_Pred_2.png}
    \caption{Visualization of the performance of YOLO-NAS-L with bounding boxes and corresponding obstacle
classes predicted by the model.}
    \label{fig:L-annot-pred}
\end{figure}

\begin{figure}[H]
The Confusion Matrix illustrates the effectiveness of the overall YOLO-NAS-L in detecting obstacles across four distinct classes. Notably, the matrix reveals a notable variation in the true positive values. Specifically, the highest true positive rate, standing at 0.91 which is good, is achieved in detecting the 'motorbike' class. Conversely, the 'bus' class exhibits the lowest true positive rate, measured at 0.44 which is very poor.  
    \centering
    \includegraphics[width=\linewidth]{tex/img/YNL2_CM.png}
    \caption{YOLO-NAS-L Overall results }
    \label{fig:ConfusionMatrixY-N-L}
\end{figure}



\newpage
\subsection{Analysis of Performance for YOLO-NAS}
\begin{table}[htbp] 
\centering
\caption{Over-all results of the three models from the confusion matrix}
\label{tab:Comp}
\footnotesize
\begin{tabular}{|c|c|c|c|c|c|c|c|}%{ |c|c|c| }
\hline
\multicolumn{ 6}{|c|}{\textbf{YOLO-NAS-SMALL}}\\
\hline
\hline
\multirow{19}[1]{*}{True}
  & & Bus& Car& Motorbike& Truck \\
\hline
& Bus& \cellcolor[HTML]{00FFFF}{0.71}& & &  \\
\hline
& Car& & \cellcolor[HTML]{00FFFF}{0.93}& &  \\
\hline
& Motorbike& & & \cellcolor[HTML]{00FFFF}{1}&  \\
\hline
& Truck& & & & \cellcolor[HTML]{00FFFF}{0.55} \\ 
\hline
\hline
\multicolumn{ 6}{|c|}{\textbf{YOLO-NAS-MEDIUM}}\\
\hline
\hline
  & & Bus& Car& Motorbike& Truck \\
\hline
& Bus& \cellcolor[HTML]{00FFFF}{0.83}& & &  \\
\hline
& Car& & \cellcolor[HTML]{00FFFF}{0.87}& &  \\
\hline
& Motorbike& & & \cellcolor[HTML]{00FFFF}{0.91}&  \\
\hline
& Truck& & & & \cellcolor[HTML]{00FFFF}{0.56} \\ 
\hline
\hline
\multicolumn{ 6}{|c|}{\textbf{YOLO-NAS-LARGE}}\\
\hline
\hline
  & & Bus& Car& Motorbike& Truck \\
\hline
& Bus& \cellcolor[HTML]{00FFFF}{0.44}& & &  \\
\hline
& Car& & \cellcolor[HTML]{00FFFF}{0.90}& &  \\
\hline
& Motorbike& & & \cellcolor[HTML]{00FFFF}{0.91}&  \\
\hline
& Truck& & & & \cellcolor[HTML]{00FFFF}{0.62} \\ 
\hline
\hline
&  \multicolumn{ 5}{|c|}{\textbf{Predicted}}\\
\hline
\end{tabular}
\end{table}
The primary objective of table \ref{tab:Comp} is to provide a comprehensive analysis of the True Positive (TP) rate of three different models upon completion of their training. The table presents a comparative evaluation of the models' performance in detecting various classes, thus enabling researchers to identify the model that performs best. It is evident from the results that YOLO-NAS-M outperforms the other models in detecting all classes. Therefore, researchers can also fine-tune this model for future applications that require high accuracy in detecting various classes, similar to the training dataset mentioned in \ref{datset}.


%%%%%%%%%%%%%%%%%
\begin{table}[!htbp]
    \centering
    \caption{Analysis of Performance for YOLO-NAS}
    \label{tab:my_label}
    \footnotesize
    \begin{tabular}{|c|c|c|c|c|c|c|c|c|}
    \hline
     Models &  Loss\_cls&  Loss\_iou& Loss\_dfl & Loss &  Precision@0.5&  Recall@0.5&  mAP@0.5& F1@0.5\\
    \hline
    NAS-Small&  0.638&  0.122&  0.729&  1.308&  0.0717&  0.9327&  0.7812& 0.1323\\
    \hline
    NAS-Medium& 0.6198 &  0.1194&  0.7262&  1.2814& 0.0831 &  0.9308 & 0.8339 & 0.1513\\
    \hline
    NAS-Large& 0.6000 &  0.1154&  0.7355& 1.256 &  0.098&  0.9813&  0.8707& 0.1779\\
         \hline
    \end{tabular}
\end{table}

The performance metrics presented in Table \ref{tab:my_label} highlight interesting facets of the behavior of the different YOLO-NAS variants. First, the Recall scores for all models are exceptionally high, above 0.93 in all cases, which means that these models are very good at detecting the classes. This could be crucial in scenarios where the priority is to detect as many positive cases as possible, even at the risk of detecting some false classes.


Conversely, the precision scores present a contrasting narrative. The scores for all models are significantly lower, particularly in comparison to the high recall values. YOLO-NAS-L, the largest model, offers the best Precision at 0.098, YOLO-NAS-M at 0.0831, and YOLO-NASs at 0.0717. These low Precision scores suggest that while the models are good at capturing positive cases, they also have a high rate of false positives, this false positive is mostly coming from one class that is 'car' We can clearly see that in figure \ref{fig:ConfusionMatrixY-N-S}, \ref{fig:ConfusionMatrixY-N-M} \ref{fig:ConfusionMatrixY-N-L}

The F1-score, grows with the model size, reflecting that larger models achieve a better balance between these two measures.

The mAP@0.50 scores show that model size matters in delivering the best overall performance and that the larger models deliver the best overall performance.\\

In summary, while all the YOLO-NAS models exhibit
a strong ability to capture positive cases, they struggle with precision, indicating high false positives. Though marginally, the models’ complexity seems to positively influence precision and F1-score, suggesting potential benefits of
using larger models if computational resources and training time permit. However, the precision-recall trade-off must be carefully considered based on the specific requirements of the detection task.

\section{Summary}
The paper explores the use of Neural Architecture Search (AutoNAC) to optimize YOLO-NAS for latency and throughput, resulting in three architectures (YOLO-NASS, YOLO-NASM, and YOLO-NASL). Fine-tuning on a custom dataset is performed, and the model's performance is evaluated on the RoboFlow100 dataset using metrics like IoU, Average Precision, mAP, Precision, Recall, and F1 Score. YOLO-NAS is found to be efficient, adaptable to diverse tasks and hardware, accurate in detection, and well-suited for applications like traffic management, security, and autonomous vehicles.
In conclusion, the study presents a comprehensive exploration of deep learning, object detection algorithms, and the YOLO-NAS model's application in real-time object recognition for autonomous vehicles, if we have reasonable dataset.
 % Można też pisać rozdziały w jednym pliku.
\clearpage % Zawsze zaczynamy rozdział od nowej strony
% \lipsum[4-10]

%---------------
% Bibliografia
%---------------
\cleardoublepage % Zaczynamy od nieparzystej strony
\printbibliography
\clearpage

% Wykaz symboli i skrótów.
% Pamiętaj, żeby posortować symbole alfabetycznie
% we własnym zakresie. Makro \acronymlist
% generuje właściwy tytuł sekcji, w zależności od języka.
% Makro \acronym dodaje skrót/symbol do listy,
% zapewniając podstawowe formatowanie.
\acronymlist
\acronym{COCO}{Common Objects in Context}
\acronym{DNN}{Deep Neural Networks}
\acronym{ANN}{Artificial Neural Network }
\acronym{MLP}{Multilayer Perceptron}
\acronym{ReLU}{Rectified Linear Unit}
\acronym{SELU}{Scaled Exponential Linear Unit }
\acronym{MSE}{Mean Squared Error}
\acronym{MAE}{Mean Absolute Error}
\acronym{MSLE}{Mean Squared Logarithmic Error}
\acronym{SVM}{Support Vector Machine}
\acronym{KLD}{Kullback-Leibler Divergence}
\acronym{CNN}{Convolutional Neural Network}
\acronym{RMSProp}{Root Mean Square Prop}
\acronym{SGD}{Stochastic Gradient Descent}
\acronym{GRU}{Gated Recurrent Unit}
\acronym{LSTM}{Long-Short term Memory}
\acronym{RNN}{Recurrent Neural Network}
\acronym{BPTT}{Back-Propagation Through Time}
\acronym{SOM}{Self Organized Map}
\acronym{RBM}{Restricted Boltzmann Machines }
\acronym{DBN}{Deep Belief Networks}
\acronym{DNS}{Deep Stacking Networks}
\acronym{SOTA}{State-of-Art}
\acronym{YOLO}{You Look Only Once}
\acronym{R-CNN}{Region-Based Convolutional Neural Network}
\acronym{RPN}{Region Proposal Network}
\acronym{FPN}{Feature Pyramid Network}
\acronym{SSD}{Single Shot Detection}
\acronym{IoU}{Intersection Over Union}
\acronym{mAP}{Mean of Average Precision}
\acronym{VOC}{Visual Object Classes}
\acronym{TP}{True Positives}
\acronym{TN}{True Negatives}
\acronym{FP}{False Positives}
\acronym{FN}{False Negatives}
\acronym{VGG}{Visual Geometry Group}
\acronym{PP-YOLO}{Paddle Paddle \emph{You Look Only Once}}
\acronym{MAE-NAS}{Maximum Entropy  Neural Architecture Search}
\acronym{GPU}{Graphics Processing Unit}
\acronym{E-ELAN}{Extended Efficient Layer Aggregation Network}
\acronym{ET-head}{Efficient Task-Aligned Head}
\acronym{VFL}{Varifocal}
\acronym{DFL}{Distribution Focal Loss}
\acronym{FPS}{Frames Per Second}
\acronym{PAN}{Path Aggregation Network}
\acronym{YOLO-NAS}{\emph{You Look Only Once Neural Architectural Search}}
\acronym{KD}{Knowledge Distillation}
\acronym{AutoNAC}{Automated Neural Architecture Construction}
\acronym{QSP}{Quantization-Specific Parameters}
\acronym{QCI}{Quantization-Centric Initialization}
\acronym{NAS}{Neural Architectural Search}
\acronym{SPP}{Spatial Pyramid Pooling}
\acronym{AI}{Artificail Intelligence}
\acronym{YOLO-NASS}{YOLO-NAS-Small}
\acronym{YOLO-NASM}{YOLO-NAS-Medium}
\acronym{YOLO-NASL}{YOLO-NAS-Large}
\acronym{RF}{Roboflow}
\acronym{EMA}{Exponential Moving Average }
\acronym{AP}{Average Precision}
\acronym{SGLogger}{Super-Gradient Logger}
\vspace{0.8cm}

%--------------------------------------
% Spisy: rysunków, tabel, załączników
%--------------------------------------
\pagestyle{plain}

\listoffigurestoc    % Spis rysunków.
\vspace{1cm}         % vertical space
\listoftablestoc     % Spis tabel.
\vspace{1cm}         % vertical space

%I commented out the appenecis
%\listofappendicestoc % Spis załączników

%-------------
% Załączniki
%-------------

% Obrazki i tabele w załącznikach nie trafiają do spisów
% \captionsetup[figure]{list=no}
% \captionsetup[table]{list=no}

% % Załącznik 1
% \clearpage
% \appendix{Nazwa załącznika 1}
% % \lipsum[1-3]
% \begin{figure}[!h]
% 	\centering \includegraphics[width=0.5\linewidth]{logopw2.png}
% 	\caption{Obrazek w załączniku.}
% \end{figure}
% % \lipsum[4-7]

% % Załącznik 2
% \clearpage
% \appendix{Nazwa załącznika 2}
% % \lipsum[1-2]
% \begin{table}[!h] \centering
%     \caption{Tabela w załączniku.}
%     \begin{tabular} {| c | c | r |} \hline
%         Kolumna 1       & Kolumna 2 & Liczba \\ \hline\hline
%         cell1           & cell2     & 60     \\ \hline
%         \multicolumn{2}{|r|}{Suma:} & 123,45 \\ \hline
%     \end{tabular}
% \end{table}
% \lipsum[3-4]

% Używając powyższych spisów jako szablonu,
% możesz dodać również swój własny wykaz,
% np. spis algorytmów.

\end{document} % Dobranoc.
