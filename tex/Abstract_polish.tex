Streszczenie. Celem pracy jest analiza systemu unikania przeszkód przez pojazdy autonomiczne, w tym roboty mobilne, opartego na metodach głębokiego uczeniu się złożonych algorytmów sztucznych sieci neuronowych. Wykorzystanie zaawansowanych technik wizji komputerowego w wyposażeniu pojazdów autonomicznych, zapewnia zdolność postrzegania i analizy otoczenia, wykrywania przeszkód w czasie rzeczywistym i bezpiecznego poruszania się w złożonym otoczeniu. Praca przedstawia ocenę wydajności modelu sieci YOLO-NAS i jego wariantów, podkreślając jego wysoką wydajność w zakresie osiągania optymalnej równowagi pomiędzy dokładnością i szybkością przy użyciu tych samych hiperparametrów, w czym przewyższa inne modele zaprojektowane przez człowieka. Przeprowadzona analiza dodatkowo podkreśla możliwość dostosowania modelu YOLO-NAS do różnorodnych zadań i sprzętu, pozycjonując go jako cenne narzędzie o szerokim spektrum zastosowań. Strategia wstępnego treningu YOLO-NAS wyposaża ten model w „umiejętność” wszechstronnego zrozumienie rozwiązywanego zadania, zwiększając jego możliwości w zastosowaniach do wykrywania obiektów w czasie rzeczywistym.
