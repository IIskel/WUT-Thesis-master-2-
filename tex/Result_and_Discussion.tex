\section{Result and discussion}
The YOLO-NAS has YOLO-NAS-L, YOLO-NAS-M, and YOLO-NAS-S variants as we have discussed earlier and the result and discussion are tailored to individual performance characteristics of each model, and every model has three paths to choose from; the best model, average model and latest model, and particularly  the inference performed after the training were used best path of each model.
\subsection{Dataset and Preprocessing}
In section [\nameref{datset}] I have discussed the dataset properties
\begin{figure}[H]
    \centering
    \includegraphics[width=\linewidth]{tex/img/sample_dataset.jpg}
    \caption{sample image for all model}
    \label{fig:sample_dataset}
\end{figure}
%%%%%%%%%%%%%%%%%%%%%%%%%%%%%
%YOLO_NAS_S
%%%%%%%%%%%%%%%%%%%
\begin{figure}[H]
  \begin{minipage}{0.48\textwidth}
    \centering
    \includegraphics[width=\linewidth]{tex/img/S-before_trainingYN_S.png}
    \caption{YOLO-NAS-S: Inference 
    on the Model before training}
    \label{fig:YOLO-NASSM_vs_other_models}
  \end{minipage}%
  \begin{minipage}{0.5\textwidth}
    \centering
    \includegraphics[width=\linewidth]{tex/img/S-After training_YN_S.png}
    \caption{YOLO-NAS-S: Inference on the Model after training}
    \label{fig:YOLO-NAS_vs_other_models}
  \end{minipage}
  \caption{YOLO-NAS: Inference on the model before and  after training .}
\end{figure}

\begin{figure}[H]
    \centering
    \includegraphics[width=0.6\linewidth]{tex/img/S-annotation_&_prediction.png}
    \caption{Annotation and Prediction}
    \label{fig:L-annot-pred}
\end{figure}

\begin{figure}[H]
    \centering
    \includegraphics[width=\linewidth]{tex/img/S-Overall_or_confusionMatrix.png}
    \caption{YOLO-NAS-S Overall results }
    \label{fig:ConfusionMatrixY-N-S}
\end{figure}
%%%%%%%%%%%%%%%%%%%%%%%%
%YOLONAS-M
%%%%%%%%%%%%%%%%%%%%
\begin{figure}[H]
  \begin{minipage}{0.48\textwidth}
    \centering
    \includegraphics[width=\linewidth]{tex/img/M-before_training.png}
    \caption{YOLO-NAS-M: Inference 
    on the Model before training}
    \label{fig:YOLO-NASSM_vs_other_models}
  \end{minipage}%
  \begin{minipage}{0.5\textwidth}
    \centering
    \includegraphics[width=\linewidth]{tex/img/M-after training.png}
    \caption{YOLO-NAS-M: Inference on the Model after training}
    \label{fig:YOLO-NAS_vs_other_models}
  \end{minipage}
  \caption{YOLO-NAS-M: Inference on the model before and  after training .}
\end{figure}

\begin{figure}[H]
    \centering
    \includegraphics[width=0.6\linewidth]{tex/img/M-Annotation_&_Predictions.png}
    \caption{Annotation and Prediction}
    \label{fig:L-annot-pred}
\end{figure}

\begin{figure}[H]
    \centering
    \includegraphics[width=\linewidth]{tex/img/M-confiusion.png}
    \caption{YOLO-NAS-M Overall results }
    \label{fig:ConfusionMatrixY-N-S}
\end{figure}

%%%%%%%%%%%%%%%%%%%%%%%%%%%%%%
%YOLO-NAS-L
%%%%%%%%%%%%%%%%%%%%%%%%%%
\begin{figure}[H]
  \begin{minipage}{0.48\textwidth}
    \centering
    \includegraphics[width=\linewidth]{tex/img/L-before_training.png}
    \caption{YOLO-NAS-L: Inference 
    on the Model before training}
    \label{fig:YOLO-NASSL_vs_other_models}
  \end{minipage}%
  \begin{minipage}{0.5\textwidth}
    \centering
    \includegraphics[width=\linewidth]{tex/img/L-after_training.png}
    \caption{YOLO-NAS-L: Inference on the Model after training}
    \label{fig:YOLO-NAS_vs_other_models}
  \end{minipage}
  \caption{YOLO-NAS-L: Inference on the model before and  after training .}
\end{figure}


\begin{figure}[H]
    \centering
    \includegraphics[width=0.6\linewidth]{tex/img/L-annotations_predictions.png}
    \caption{Annotation and Prediction}
    \label{fig:L-annot-pred}
\end{figure}

\begin{figure}[H]
    \centering
    \includegraphics[width=\linewidth]{tex/img/L-confusion_matrix.png}
    \caption{YOLO-NAS-L Overall results }
    \label{fig:ConfusionMatrixY-N-S}
\end{figure}

\begin{table}[htbp]
\centering
\caption{Over-all results of the three model}
\footnotesize
\begin{tabular}{|c|c|c|c|c|c|c|c|c|}%{ |c|c|c| }
\hline
\multicolumn{ 6}{|c|}{\textbf{YOLO-NAS-SMALL}}\\
\hline
\hline
 & Bus& Car& Motorbike& Truck& van \\
\hline
Bus& \cellcolor[HTML]{00FFFF}{0.23}& & & & \\
\hline
Car& & \cellcolor[HTML]{00FFFF}{0.77}& & & \\
\hline
Motorbike& & & \cellcolor[HTML]{00FFFF}{0.75}& & \\
\hline
Truck& & & & \cellcolor[HTML]{00FFFF}{0.73}& \\ 
\hline
Van& & & & & \cellcolor[HTML]{00FFFF}{0.68}\\
\hline
\hline
\multicolumn{ 6}{|c|}{\textbf{YOLO-NAS-MEDIUM}} \\ \hline
\hline
Bus& \cellcolor[HTML]{00FFFF}{0.37}& & & & \\
\hline
Car& & \cellcolor[HTML]{00FFFF}{0.81}& & & \\
\hline
Motorbike& & & \cellcolor[HTML]{00FFFF}{1.00}& & \\
\hline
Truck& & & & \cellcolor[HTML]{00FFFF}{0.78}& \\ 
\hline
Van& & & & & \cellcolor[HTML]{00FFFF}{0.67}\\
\hline
\hline
\multicolumn{ 6}{|c|}{\textbf{YOLO-NAS-LARGE}} \\ \hline
\hline
Bus& \cellcolor[HTML]{00FFFF}{0.69}& & & & \\
\hline
Car& &\cellcolor[HTML]{00FFFF}{0.86} & & & \\
\hline
Motorbike& & & \cellcolor[HTML]{00FFFF}{0.75} & & \\
\hline
Truck& & & &\cellcolor[HTML]{00FFFF}{0.72} & \\ 
\hline
Van& & & & &\cellcolor[HTML]{00FFFF}{0.63} \\
\hline
\end{tabular}
\end{table}


\subsection{Training and Implementation Details}
For all of the models, I’ve used a batch size of 16, and trained for 20 epochs using NVIDIA Tesla T4 GPU. I implemented and trained the models using Google Colab. 

I have tweaked various hyperparameters for this work. I have set the ”average best models” to true, warm-up models linear epoch step, initial learning rate for warm-up as 1e-6, learning rate decay factor during warmup epochs as 3, initial learning rate as 5e-4, learning rate decay mode as cosine, cosine-final learning rate ratio as 0.1, optimizer as Adam, weight decay in optimizer parameters as 0.0001. I have used zero weight decay on bias and batch-normalization and leveraged exponential moving averaging with a decay factor of 0.9 and decay type as a threshold with the” mixed precision” set to true.
\newpage
\subsection{Analysis of Performance for YOLO-NAS}

\begin{table}[!htbp]
    \centering
    \caption{Analysis of Performance for YOLO-NAS}
    \label{tab:my_label}
    \footnotesize
    \begin{tabular}{|c|c|c|c|c|c|c|c|c|}
    \hline
     Models &  Loss\_cls&  Loss\_iou& Loss\_dfl & Loss &  Precision@0.5&  Recall@0.5&  mAP@0.5& F1@0.5\\
    \hline
    NAS-Small&  0.682&  0.1709&  0.7117&  1.465&  0.07896&  0.8864&  0.6792& 0.14077\\
    \hline
    NAS-Medium& 0.67099 &  0.1688&  0.7356&  1.4609& 0.0917 &  0.9114 & 0.7007 & 0.1596\\
    \hline
    NAS-Large& 0.6535 &  0.1617&  0.7189& 1.4172 &  0.11563&  0.9116&  0.7269& 0.19652\\
         \hline
    \end{tabular}
\end{table}

The performance metrics presented in Table \ref{tab:my_label} highlight interesting facets of the behavior of the different YOLO-NAS variants. First, it is worth mentioning that the Recall scores for all models are exceptionally high, above 0.88 in all cases, which means that these models are very good at detecting the classes. This could be crucial in scenarios where the priority is to detect as many positive cases as possible, even at the risk of detecting some false classes.


The Precision scores, however, tell a different story. They are markedly lower for all models, especially when compared to the high Recall scores. YOLO-NASl, the largest model, offers the best Precision at 0.11563, followed by YOLO-NASm at 0.0917, and finally, YOLO-NASs at 0.079. These low Precision scores suggest that while the models are good at capturing positive cases, they also have a high rate of false positives. 

The F1-score, grows with the model size, reflecting that larger models achieve a better balance between these two measures.

Regarding, the mAP@50 scores, also grow with the model size, reflecting that the larger models deliver a slight edge in overall performance.\\

In summary, while all the YOLO-NAS models exhibit
strong ability in capturing positive cases, they struggle with precision, indicating a high level of false positives. The models’ complexity seems to positively influence precision and F1-score, though marginally, suggesting potential benefits of
using larger models if computational resources and training time permit. However, the precision-recall trade-off must be carefully considered based on the specific requirements of the detection task.
