\section{Summary}
The paper discusses the implementation of Neural Architecture Search (AutoNAC) to optimize YOLO-NAS for latency and throughput, resulting in three new architectures (YOLO-NASS, YOLO-NASM, and YOLO-NASL). The models were fine-tuned on a specific dataset from the Roboflow universe. Their performance was evaluated on a separate dataset using various performance metrics such as Intersection over Union (IoU), Mean Average Precision (mAP), Precision, Recall, and F1 Score.

After extensive training, notably, YOLO-NAS-S achieved a mAP@0.5 of 0.7812, an inference time of 5.38e+3, a recall of 0.9327, and an F1 Score of 0.1323. YOLO-NAS-M attained a mAP@0.5 of 0.8339, an inference time of 7.78e+3, a recall of 0.9308, and an F1 Score of 0.1513. Lastly, YOLO-NAS-L demonstrated a mAP@0.5 of 0.8707, an inference time of 9.31e+3, a recall of 0.9327, and an F1 Score of 0.1779.

Based on the results, it is recommended to utilize YOLO-NAS-M for object detection tasks as it strikes a balance between accuracy and efficiency. YOLO-NAS-M is found to be efficient for real-time applications while maintaining good performance in identifying and localizing objects in images.

Generally, this model is good where we mostly focus on producing or capturing the more positive cases, and this will sometimes over-predict or may detect when it is not. 

It would be better in the future to fine-tune a big dataset that has been correctly labeled and purposefully collected for the special job to see if it will balance the precision and recall or may result in a good F1 Score.