\section{Summary}
The paper explores the use of Neural Architecture Search (AutoNAC) to optimize YOLO-NAS for latency and throughput, resulting in three architectures (YOLO-NASS, YOLO-NASM, and YOLO-NASL). Fine-tuning on a custom dataset is performed, and the model's performance is evaluated on the RoboFlow100 dataset using metrics like IoU, Average Precision, mAP, Precision, Recall, and F1 Score. YOLO-NAS is found to be efficient, adaptable to diverse tasks and hardware, accurate in detection, and well-suited for applications like traffic management, security, and autonomous vehicles.
In conclusion, the study comprehensively explores deep learning, object detection algorithms, and the YOLO-NAS model's application in real-time object recognition for autonomous vehicles if we have a reasonable dataset.


The document delves into developing a deep learning-based obstacle avoidance system for mobile robots utilizing Neural Architecture Search (NAS) in Python. It emphasizes the goal of enabling mobile robots to perceive their environment, identify obstacles in real time, and navigate through intricate surroundings securely and expediently. The study incorporates a combination of top-down and bottom-up information flows via a highly advanced pyramid-attention neck to effectively capture and leverage multi-scale information, thus enhancing the network's capacity to detect objects of varying sizes and complexities. Additionally, it highlights the backbone of YOLO-NAS as a pivotal component in its architecture. It was developed through an advanced Network Architecture Search process using Deci's proprietary NAS technology, AutoNAC, to tailor the network structure with unparalleled precision for object detection tasks. 